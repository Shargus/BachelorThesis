%% INIZIO SORGENTE
\documentclass[a4paper,12pt,oneside]{book}

%% pacchetti usati:
% babel, indent first, fancy headers, amssymb, amsmath, latexsym, geometry per le specifiche, parindent=1cm, frontespizio (!)

\usepackage[english,italian]{babel}
\usepackage[utf8]{inputenc}
\usepackage[T1]{fontenc}
\usepackage{indentfirst}
\usepackage{fancyhdr}
\usepackage[titletoc]{appendix} %appendici

% per far vedere le etichette, da togliere quando si deve stampare
%\usepackage{showkeys}

% per la matematica
\usepackage{amsmath}
\usepackage{latexsym}
\usepackage{amssymb}
\usepackage{mathtools}
\usepackage{stackengine}

% comandi utili
\newcommand{\eqdef}{\stackrel{\mathclap{\normalfont\mbox{def}}}{=}}
\newcommand\oast{\stackMath\mathbin{\stackinset{c}{0ex}{c}{0ex}{\ast}{\bigcirc}}}
\newcommand{\R}{\mathbb{R}}
\newcommand{\N}{\mathbb{N}}
\newcommand{\Z}{\mathbb{N}}
\newcommand{\ii}{(i)}
\newcommand{\XX}{\mathbf{X}}
\newcommand{\WW}{\mathbf{W}}
\newcommand{\bb}{\mathbf{b}}
\newcommand{\xii}{x_i}
\newcommand{\xjj}{x_j}
\newcommand{\abs}[1]{\left|#1\right|} %valore assoluto
\usepackage[top=3cm, bottom=3cm, left=3.5cm, right=2.5cm]{geometry}
\parindent=1cm
\usepackage[swapnames]{frontespizio}
\usepackage{changepage}

% per le immagini
\usepackage{graphicx}
\graphicspath{{img/}{img_esperimenti/}{img_appendice/}{img_esperimenti/foto_azzorre/}{img_esperimenti/foto_taranto/}{img_introduzione/}{img/googlenet/}{img/alexnet/}{img/resnet/}}
\usepackage{subcaption}
\usepackage{float}
\newcommand{\rulesep}{\unskip\ \vrule\ }


% per le tabelle
\usepackage{caption} %vale anche per le immagini
\captionsetup{aboveskip=10pt}
\captionsetup{belowskip=0pt}
\captionsetup[table]{position=bottom}
\usepackage{tabularx, booktabs}
\newcommand{\acapo}[1]{%
  \begin{tabular}{@{}c@{}}\strut#1\strut\end{tabular}%
}
\usepackage{adjustbox}

% per il codice sorgente
\usepackage{verbatim}

% per la bibliografia
\usepackage[autostyle, italian=guillemets]{csquotes}
\usepackage[backend=biber, style=numeric-comp, babel=hyphen, sorting=none]{biblatex}
\usepackage{url}
\usepackage{fancyvrb}
\addbibresource{biblio_tesi.bib}


% DEFINIZIONE AMBIENTE ABSTRACT
\newenvironment{abstract}%
{\cleardoublepage%
\thispagestyle{empty}%
\null \vfill\begin{center}%
\bfseries \abstractname \end{center}}%
{\vfill\null}

%% INIZIO TESI
\frontmatter
\pagestyle{empty}

\begin{document}

%% CREAZIONE FRONTESPIZIO
\begin{frontespizio}
\Istituzione{POLITECNICO DI BARI}
\Logo[5cm]{logo}
\Dipartimento{Ingegneria Elettrica e dell'Informazione - DEI}
\Corso[Laurea Triennale]{Ingegneria Informatica e dell'Automazione (D.M. 270/04)}
\Titolo{ANALISI DI IMMAGINI CON RETI NEURALI CONVOLUZIONALI PER LA CLASSIFICAZIONE DEI CETACEI NEL GOLFO DI TARANTO}
\Titoletto{Tesi di laurea in CALCOLO NUMERICO}
\Candidato[568581]{Tommaso MONOPOLI}
\NCandidato{Laureando}
\Relatore{Chiar.mo Prof. Ing. Tiziano POLITI}
\Correlatore{Dott. Ing. Vito RENÒ}
\Annoaccademico{2018-2019}
\Rientro{1.5cm}
\Preambolo{\renewcommand{\frontlogosep}{10pt}}
\end{frontespizio}

%% SOMMARIO
\begin{abstract}
Il sistema terrestre è sempre stato soggetto alle conseguenze delle attività umane, e la biodiversità degli ecosistemi acquatici e marini è fortemente a rischio. Diversi studi cercano di capire in che modo la perdita della biodiversità possa alterare l’integrità e il funzionamento di tali ecosistemi.  Una risposta a questa domanda può essere ricercata negli studi effettuati sulla distribuzione e sullo stato di conservazione dei cetacei, oggetto di numerose ricerche negli ultimi anni.\\
Un'attività mirata alla raccolta di informazioni rilevanti allo studio dei cetacei è la \textit{foto-identificazione degli individui} di una specie, che prevede il riconoscimento - automatico o manuale - di uno stesso individuo in diverse immagini collezionate nel tempo, mediante l’analisi di particolari segni distintivi (\emph{feature}) presenti nell'immagine.\\
Questa attività può essere effettuata manualmente, ma con un grande costo in termini di tempo per i ricercatori, che spesso hanno a disposizione diverse migliaia o milioni di fotografie, scattate nel corso di anni. L'evidente difficoltà nell'approccio manuale alla foto-identificazione dei cetacei (tutt'oggi ancora ampiamente operata) suggerisce l’applicazione di metodologie di \emph{Computer Vision} per automatizzare tale attività.
L’obbiettivo del presente lavoro di tesi è la creazione di classificatori binari che ricevano in input un dataset di immagini bidimensionali collezionate nei pressi delle isole Azzorre (Oceano Atlantico settentrionale) e sappiano suddividere lo stesso dataset in due classi di immagini, a seconda che in ciascuna immagine sia rilevata o meno una \emph{feature} utile ad una successiva foto-identificazione. Nel caso dei cetacei, il criterio di classificazione è la presenza nell'immagine della pinna dorsale dell'individuo.\\
Le metodologie impiegate sono quelle del \emph{machine learning}; in particolare, si è scelto di utilizzare la tecnica del \emph{transfer learning} per il riuso e il ri-adattamento di modelli pre-addestrati, usati per risolvere task di classificazione diversi da quello in esame.
Gli esperimenti condotti su dati reali acquisiti in mare dimostrano l’utilità di tali tecniche di Computer Vision nel campo della foto-identificazione dei cetacei.
\end{abstract}

%% INDICE GENERALE
\tableofcontents
%\listoffigures
%\listoftables

%%
%% --- MATERIALE PRINCIPALE ---
%%
\mainmatter

%% INTRODUZIONE
\chapter{Introduzione}
\label{introduzione}
%\addcontentsline{toc}{chapter}{Introduzione}
La foto-identificazione è una tecnica largamente impiegata per l’identificazione dei singoli individui a partire da una o più immagini. Il principale vantaggio di questa tecnica è la sua non invasività che la rende particolarmente utile per studiare sia la dinamicità che i movimenti di ogni specie. Tale metodologia risulta essere uno strumento affidabile quando viene applicato nella comprensione dei comportamenti dei cetacei (migrazioni e spostamenti). Tra i delfini, vi sono due specie più adatte a tali studi.  Il primo delfino riguarda la specie “Tursiops truncatus” (tursiope) o delfino dal naso a bottiglia, mentre la seconda specie riguarda la specie “Grampus griseus” (Grampo, o delfino di Risso), avente numerose cicatrici su tutto il corpo, entrambi appartenenti alla famiglia dei Delfinidi.[1]

\section{Problema e obiettivi}
L'obiettivo principale del presente lavoro di tesi è stato quello di creare un sistema per il rilevamento automatico della presenza di cetacei all'interno di uno scatto fotografico.

Lo scopo finale è facilitare lo studio dei cetacei, attorno al quale si riunisce grande interesse scientifico (par. \ref{TODO}), incentivando l'utilizzo di tecniche non invasive basate su algoritmi innovativi e grandi disponibilità di dati. Uno dei principali metodi di studio non-invasivi dei cetacei è la foto-identificazione degli esemplari (par. \ref{fotoidentificazione}). Il presente lavoro di tesi rappresenta un passo avanti verso la completa automatizzazione del processo di foto-identificazione degli esemplari incontrati durante le campagne di avvistamento, fornendo un miglioramento nelle prestazioni di una routine (CropFin v1, par. \ref{cropFin}) che può aiutare i biologi nel successivo lavoro di foto-identificazione.

Il problema affrontato TODO

%% CAPITOLI
\chapter{Metodologie}\label{teoria}
\pagestyle{fancy}
\fancyhf{}
\fancyhead[OL]{\rightmark}
\cfoot{\thepage}

Nel corso di questo capitolo saranno introdotti progressivamente i concetti principali su cui si fonda la \textit{computer vision} (visione artificiale) e il \textit{machine learning} (apprendimento automatico). Questi presupposti teorici permetteranno di comprendere da un punto di vista teorico quanto descritto nella sezione sperimentale della tesi (cap. \ref{esperimenti}.

Per i primi paragrafi, dedicati all'\textit{image processing}, le principali fonti seguite sono \cite{gianvito} e \cite{computerVision}.

Per i paragrafi dedicati al \textit{machine} e \textit{deep learning} le fonti di riferimento sono \cite{dlbook} e soprattutto \cite{cs231n}.


\section{Immagini digitali}
Caratterizziamo intuitivamente il concetto di "immagine" dal punto di vista informatico.

Un'\textbf{immagine digitale} è una rappresentazione binaria di un'immagine (in generale a colori) a due dimensioni\footnote{Ci riferiamo in questa sede solo alle immagini di tipo raster, quelle cioè con risoluzione e numero di canali di colore fissati a priori, come ad esempio le immagini digitali in formato jpg.}; essa può essere definita matematicamente come un tensore $\mathcal{I}\in\{0,\dots,255\}^{h\times w\times c}$, dove $h$ e $w$ sono rispettivamente dette \textbf{altezza} e \textbf{larghezza} dell'immagine, la coppia $(w,h)$ \textbf{risoluzione} mentre $c$ è il numero di \emph{canali di colore}\footnote{Spesso si scrive che $\mathcal{I}$ è un immagine $w\times h\times c$, o più semplicemente $w\times h$ (assumendo $c=3$)}. Nello \textbf{spazio di colore sRGB} (standard RGB, d'ora in avanti abbreviato in RGB), ampiamente adoperato, i canali di colore sono rosso (R, Red), verde (G, Green) e blu (B, Blue), quindi $c=3$. In mancanza di diverse indicazioni, ci si riferirà nel seguito allo spazio di colore RGB.

Un \textbf{pixel} $p(i,j)$ è definito come la funzione vettoriale
\[p(i,j)=[r(i,j),g(i,j),b(i,j)]\]
essendo $r,g,b:\{0,\dots,h\}\times\{0,\dots,w\}\to\{0,\dots,255\}$ le funzioni scalari che associano ad ogni posizione bidimensionale $i,j$ dell'immagine un valore intero di \textbf{intensità luminosa} compreso tra 0 e 255, uno per ciascuno dei tre canali RGB.
Ogni pixel definisce univocamente un colore nello spazio RGB, il quale può quindi rappresentare in tutto $256^{3}$ colori diversi, cioè circa 17 milioni.

Si può immaginare il tensore immagine $\mathcal{I}$ come una "pila" di tre matrici, una per ogni canale di colore, come mostrato in figura \ref{fig:rappresentazione_tensore}.

\begin{figure}[h]
\centering
\includegraphics[scale=0.7]{rappresentazione_tensore}
\caption{Rappresentazione grafica di un tensore tridimensionale; in ogni posizione compaiono gli indici del tensore}
\label{fig:rappresentazione_tensore}
\end{figure}

Un esempio di immagine digitale, scomposta nelle sue componenti RGB, è mostrato in figura \ref{fig:canali_rgb}.

\begin{figure}[h]
  \begin{minipage}[b]{0.46\textwidth}
    \includegraphics[width=\textwidth]{canali_rgb}
    \caption{Canali RGB di un'immagine}
    \label{fig:canali_rgb}
  \end{minipage}
  \hfill
  \begin{minipage}[b]{0.46\textwidth}
    \includegraphics[width=\textwidth]{pixel}
    \caption{Pixel di un'immagine}
  \end{minipage}
\end{figure}

In Matlab un'immagine digitale può essere rappresentata con il tipo di dato \verb|multidimensional-array| con tre dimensioni (corrispondenti ai tre canali di colore RGB), in cui ciascun elemento è di tipo \verb|uint8| (ma può appartenere anche ad altri tipi di dato\footnote{\url{https://it.mathworks.com/help/matlab/ref/image.html?s_tid=doc_ta#buqdlnb-C}}).

È possibile importare un'immagine RGB con la funzione:
\begin{verbatim}
im = imread("immagine.jpg");
\end{verbatim}
I canali Rosso \verb|R|, Verde \verb|G| e blu \verb|B| possono essere ottenuti come:
\begin{verbatim}
R = im(:,:,1);
G = im(:,:,2);
B = im(:,:,3);
\end{verbatim}
Il pixel di posizione $(i,j)$ è quindi:
\begin{verbatim}
p = [im(i,j,1) im(i,j,2) im(i,j,3)];
\end{verbatim}

\subsection{Spazio di colore Lab}
Oltre al modello RGB descritto al paragrafo precedente, esistono ulteriori spazi di colore che consentono di ottenere una differente rappresentazione delle medesime tonalità dei pixel.
Nel presente lavoro di tesi (par. \ref{faseRitaglio}) si utilizza una conversione delle immagini dallo spazio di colore iniziale RGB allo spazio di colore \textbf{CIE 1976 L*a*b*} (nel seguito abbreviato come Lab).
Nello spazio Lab le tonalità sono ancora espresse da triplette di valori (L*, a* e b*), ma con un significato diverso rispetto a RGB: il valore L* rappresenta la luminanza
(variazione di luminosità), mentre le altre due rappresentano la crominanza (variazione
di colore), rispettivamente una scala verde-rosso (a*) ed una scala blu-giallo (b*).

A partire dall’immagine \verb|im| è possibile convertire lo spazio di colori da RGB a Lab, ottenendo una nuova immagine \verb|im_lab|:
\begin{verbatim}
im_lab = rgb2lab(im);
\end{verbatim}
I dettagli sulla conversione di spazio di colore possono essere letti al par. 2.1.1 di \cite{gianvito}.

\subsection{Collezioni di immagini in Matlab}
Una delle necessità principali del lavoro affrontato è stata la gestione di grandi quantità di immagini. Si riportano i tipi di dato messi a dispozione da Matlab e utilizzati negli esperimenti del presente lavoro di tesi:

\begin{itemize}

\item \verb|imageDatastore|: oggetto progettato appositamente per gestire ed elaborare rapidamente una grande quantità di immagini. Per istanziare un oggetto \verb|imageDatastore| bisogna specificare l'argomento \verb|path| che indica il percorso della collezione di immagini da importare. Altri argomenti (coppie argomento-valore) opzionali per l'inizializzazione di questo oggetto sono:
\begin{itemize}
\item \verb|’IncludeSubfolders’,true|\\
Include le immagini contenute nelle sottocartelle di \verb|path| 
\item \verb|’LabelSource’,’foldernames’|\\
Assegna a ciascuna immagine un’etichetta
data dal nome della cartella in cui è contenuta
\end{itemize}
Quindi, per creare l'oggetto:
\begin{verbatim}
imds = imageDatastore(path,’IncludeSubfolders’,true,...
’LabelSource’,’foldernames’);
\end{verbatim}
L’elenco delle immagini è restituito nel campo \verb|imds.Files|.

\item \verb|augmentedImageDatastore|: oggetto creato a partire da un \verb|imageDatastore| applicando operazioni di preprocessing specificate in un oggetto \verb|imageDataAugmenter|. La sintassi di questo oggetto verrà approfondita nel par. \ref{augmentation}. TODO

\end{itemize}

\section{Trasformazioni}
Si riportano le operazioni fondamentali che hanno consentito, nel presente lavoro di tesi, di trasformare le immagini in una forma maggiormente adatta ad un’analisi successiva, una strategia chiamata \textit{image augmentation} (par. \ref{augmentation}).

\subsection{Ridimensionamento}
Il ridimensionamento consente, in generale, di ottenere una nuova immagine a risoluzione differente, riducendo o aumentando il numero di pixel utilizzati per rappresentarla.
Nel caso del presente lavoro, le dimensioni delle immagini sono state ridotte (par. \ref{faseRitaglio}), per diminuire il costo computazionale delle operazioni successive.
Per ridimensionare un'immagine \verb|im| in Matlab si può usare la funzione \verb|imresize|\footnote{La funzione imresize attua, per lo scopo, una tecnica avanzata di calcolo numerico: l’interpolazione bicubica. (TODO ? par. 2.2.1 di \cite{gianvito} per i dettagli.)}, specificando la nuova lunghezza \verb|w'| e la nuova altezza \verb|h'|:
\begin{verbatim}
im_res = imresize(im,[h' w']);
\end{verbatim}

\subsection{Rotazione}
La rotazione di un'immagine digitale può essere effettuata calcolando per ogni suo pixel $(x,y)$ il prodotto matriciale
\begin{equation*}
\begin{bmatrix}
x' \\
y'
\end{bmatrix} = 
\begin{bmatrix}
\cos\alpha & -\sin\alpha \\
\sin\alpha & \cos\alpha
\end{bmatrix}
\begin{bmatrix}
x \\
y
\end{bmatrix}
\end{equation*}
in cui $\alpha$ è l’angolo di rotazione misurato in senso antiorario rispetto all’asse x e $(x',y')$ le coordinate del pixel trasformato. L'immagine ruotata è l'insieme dei pixel $(x',y')$ calcolati.

In Matlab, la rotazione di un'immagine \verb|im| di un angolo \verb|a| può essere effettuata con
\begin{verbatim}
rotated = imrotate(im, a);
\end{verbatim}

\subsection{Riflessione}
La riflessione rispetto all’asse x di un'immagine digitale può essere effettuata calcolando per ogni suo pixel $(x,y)$ il prodotto matriciale
\begin{equation*}
\begin{bmatrix}
x' \\
y'
\end{bmatrix} = 
\begin{bmatrix}
1 & 0 \\
0 & -1
\end{bmatrix}
\begin{bmatrix}
x \\
y
\end{bmatrix}
\end{equation*}
dove $(x',y')$ sono le coordinate del pixel trasformato. L'immagine riflessa è l'insieme dei pixel $(x',y')$ calcolati.

In Matlab, la riflessione di un'immagine \verb|im| rispetto all'asse x può essere effettuata con
\begin{verbatim}
flipped = flipdim(im, 2);
\end{verbatim}

\subsection{Traslazione}
La traslazione orizzontale, a differenza delle precedenti operazioni, è una trasformazione affine ma non lineare, quindi
la rappresentazione con le matrici richiede il passaggio ad un diverso tipo di coordinate
dette omogenee:
\begin{equation*}
\begin{bmatrix}
x&y
\end{bmatrix}^\top \mapsto
\begin{bmatrix}
x&y&1
\end{bmatrix}^\top
\end{equation*}
A questo punto, è possibile ottenere le nuove coordinate traslate $(x', y')$ calcolando:

\begin{equation*}
\begin{bmatrix}
x'\\y'\\1
\end{bmatrix} = 
\begin{bmatrix}
1 & 0 & t_x \\0 & 1 & t_y \\0 & 0 & 1
\end{bmatrix}
\begin{bmatrix}
x \\ y \\ 1
\end{bmatrix}
\end{equation*}

In Matlab, la traslazione orizzontale di \verb|t| pixel di un'immagine \verb|im| può essere effettuata con
\begin{verbatim}
translated = imtranslate(im,[t 0]);
\end{verbatim}
dove \verb|t|>0 implica una traslazione verso destra, \verb|t|<0 verso sinistra.\\

I risultati delle quattro trasformazioni finora viste sono visualizzate in figura \ref{fig:trasformazioni}

\begin{figure}[h]
\centering

\begin{subfigure}[b]{\textwidth}
\centering
\includegraphics[width=0.24\textwidth]{pinnaOriginale}
\caption{Immagine originale}
\end{subfigure}

\begin{subfigure}[b]{0.24\textwidth}
\centering
\includegraphics[width=0.4\textwidth]{pinnaOriginale}
\caption{Ridimensionata}
\end{subfigure}
\begin{subfigure}[b]{0.24\textwidth}
\centering
\includegraphics[width=\textwidth]{pinnaRuotata}
\caption{Ruotata di 20$^\circ$}
\end{subfigure}
\begin{subfigure}[b]{0.24\textwidth}
\centering
\includegraphics[width=\textwidth]{pinnaRiflessa}
\caption{Riflessa}
\end{subfigure}
\begin{subfigure}[b]{0.24\textwidth}
\centering
\includegraphics[width=\textwidth]{pinnaTraslata}
\caption{Traslata di -100px}
\end{subfigure}

\caption{Trasformazioni applicate ad un'immagine digitale}
\label{fig:trasformazioni}
\end{figure}

\subsection{Convoluzione}
Nell’ambito dell’\textit{image processing} esiste una quantità notevole di operatori definiti "locali", che operano cioè non su un singolo pixel ma su un gruppo di pixel contigui.
L’operatore locale maggiormente usato è l’operatore di convoluzione. Nell'ambito del presente lavoro, esso ha un'importanza centrale per la \textit{feature extraction} delle immagini, all'interno delle reti neurali convoluzionali (par. \ref{CNN}), e più specificamente nei layer convoluzionali (par. \ref{convLayer}).


Nel seguito si presenta, pertanto, una definizione rigorosa dell'operazione di convoluzione e si forniscono semplici esempi di implementazione.

Dati due segnali discreti $x(k)$ e $w(k)$, si definisce \textbf{(somma di) convoluzione tra $x$ e $w$} una nuova funzione $s(k)$ definita come \begin{equation*}
s(k)=x(k)\ast w(k)\eqdef \sum_{i=-\infty}^{+\infty}x(i)w(k-i), i\in\Z
\end{equation*}

Si dimostra che questa operazione gode della proprietà commutativa, associativa e distributiva rispetto alla somma.

Nell'ambito della \textit{computer vision} si utilizza una versione multidimensionale dell'operazione di convoluzione, utilizzando la seguente terminologia:
\begin{itemize}
\item il primo segnale $x$ è detto \textbf{input}, generalmente costituito da un'immagine od una sua elaborazione;
\item il secondo segnale $w$ è detto \textbf{filtro} o \textbf{kernel}, solitamente costituito da una matrice
di dimensioni ridotte rispetto all'input;
\item il risultato $s$ è detto \textbf{feature map}, poiché l'operazione di filtraggio è spesso utilizzata per l'estrazione di feature a partire dall'input (par. \ref{TODO}).
\end{itemize}
È ovvio che la somma di infiniti termini prevista dalla definizione di convoluzione con segnali discreti si riduce ad una somma limitata alle dimensioni dell'immagine. Nel caso in cui l'input sia una matrice bidimensionale, ad esempio un'immagine in scala di grigi $I\in\R^{h\times w}$, anche il kernel impiegato è solitamente una matrice bidimensionale di dimensioni ridotte $K\in\R^{a\times b}$, ottenendo l’operazione:
\begin{equation*}
s(i,j)=(I\ast K)(i,j)=\sum_{k=1}^a\sum_{r=1}^b I(i+k-1,j+r-1)K(l-k+1,w-r+1)
\end{equation*}

Molte librerie, in realtà, implementano la convoluzione attraverso la funzione di crosscorrelazione (indicata con $\oast$).
In tal caso ciascun elemento della feature map si ottiene come
\begin{equation*}
s(i,j)=(I\oast K)(i,j)=\sum_{k=1}^a\sum_{r=1}^b\mathcal{I}(i+k-1,j+r-1)K(k,r)
\end{equation*}
L'operazione appena esposta - che si dimostra essere equivalente ad una convoluzione tra $I$ e $K$ - ha una semplice interpretazione, visualizzata in fig. \ref{fig:interpretazioneConvoluzione}. Convolvere un'immagine con un kernel equivale a far scorrere la matrice che rappresenta il kernel lungo l'immagine, e sviluppare i prodotti \textit{element-wise} (termine a termine) e sommarli tra loro, ottenendo l'elemento della feature map.

\begin{figure}[h]
\centering
\includegraphics[width=0.5\textwidth]{interpretazioneConvoluzione}
\caption{Un esempio di cross-correlazione (ovvero convoluzione discreta 2-D senza il
ribaltamento del kernel)}
\label{fig:interpretazioneConvoluzione}
\end{figure}

Nel caso (più comune) in cui l’input sia un tensore (ad esempio un’immagine a colori) $\mathcal{I}\in\R^{h\times w\times c}$ si richiede che il kernel abbia lo stesso numero $c$ di livelli, ad esempio $\mathcal{K}\in\R^{a\times b\times c}$. L'operazione viene così ridefinita.
\begin{equation*}
s(i,j)=(\mathcal{I}\oast\mathcal{K})(i,j,k)=\sum_{k=1}^a\sum_{r=1}^b\sum_{s=1}^c\mathcal{I}(i+s-1,j+r-1,k)K(r,s,k)
\end{equation*}
Come si nota, la feature map ottenuta sarà sempre bidimensionale (a prescindere dalla profondità $c$ del tensore in input).

Completeremo la trattazione sulla convoluzione nel par. \ref{convLayer} sul layer convoluzionale di una CNN.

\section{Problemi di \textit{Computer Vision}}
In questa sezione si riportano alcuni problemi caratteristici della \textit{computer vision}, affrontati nel corso del lavoro e finalizzati ad una comprensione di alto livello del contenuto delle immagini (e dei video) da parte del computer. Il nome italiano della disciplina, "visione artificiale", richiama in questo senso l'obiettivo di rendere artificiali i compiti svolti dal sistema visivo umano.

TODO TODO TODO vd 2.3 gianvito

\subsection{Segmentazione}
TODO
\paragraph*{Soglia di Otsu}
TODO
\paragraph*{Regioni connesse}
TODO
\paragraph*{Riempimento degli \textit{holes}}
TODO

\subsection{Object recognition}
\label{objectRecognition}
L'\textbf{\textit{object recognition}} (in italiano: riconoscimento di oggetti) nell'ambito della visione artificiale è il problema di assegnare una descrizione testuale o una o più etichette ad un'immagine, tipicamente sulla base di uno o più determinati oggetti che un computer riesce a riconoscere all'interno di essa.

Ogni categoria esistente di oggetti ha delle caratteristiche fondamentali che la differenziano da qualunque altra categoria di oggetti. Attraverso tecniche di \textit{machine learning} è possibile ricavare una descrizione di una certa categoria addestrando la macchina a riconoscere le caratteristiche (\textit{features}) fondamentali di quella categoria di oggetti, a partire da un insieme di immagini campione afferenti a quella categoria.

Per rendere affidabile il riconoscimento, è importante che l'insieme di caratteristiche estratte da ogni immagine campione sia insensibile a variazioni del punto di vista, di scala, delle condizioni di illuminazione, alle distorsioni geometriche, all'occlusione dell'oggetto, al \textit{clutter} (ingombro) di altri oggetti non informativi sullo sfondo, alle variazioni intra-classe dell'oggetto, come mostrato in fig. \ref{fig:variabilita}.

\begin{figure}[h]
\centering
\includegraphics[width=0.9\textwidth]{variabilita}
\caption{I principali "ostacoli" al riconoscimento automatico degli oggetti} 
\label{fig:variabilita}
\end{figure}

L'uomo riconosce una moltitudine di oggetti in immagini con poco sforzo, nonostante i fattori di variabilità descritti. Questo compito è ancora una sfida aperta per la computer vision in generale.\\

Il problema dell'\textbf{\textit{image classification}} è uno specifico problema di \textit{object recognition} che consiste nell'assegnare una singola etichetta (o una distribuzione di probabilità su più etichette) ad un'immagine da un insieme fisso di etichette (anche dette "categorie" o "classi"). In fig. \ref{fig:imageClassification} è visualizzato il problema in esame.

\begin{figure}[h]
\centering
\includegraphics[width=0.7\textwidth]{imageClassification}
\caption{Il task della \textit{image classification} consiste in questo caso nel calcolare una distribuzione di probabilità su quattro etichette (gatto, cane, cappello, tazza) per un'immagine digitale.} 
\label{fig:imageClassification}
\end{figure}

In letteratura sono stati proposti numerosi metodi per la risoluzione efficiente dei task di \textit{object recognition}, attraverso l'impiego di diverse tecniche di \textit{machine learning} (ad esempio l'algoritmo di clustering \textit{k-Nearest Neighbor}). Negli ultimi anni i risultati più promettenti sono stati offerti dalle \textit{reti neurali convoluzionali}, di cui parleremo diffusamente nel seguito del capitolo (par. \ref{CNN}) e che verranno impiegate negli esperimenti (par. \ref{faseClassificazione}).

\section{Object detection}
Un altro importante problema di \textit{computer vision} è l'\textbf{\textit{object detection}} (in italiano: rilevamento di oggetti). Esso consiste nella localizzazione di oggetti di categorie stabilite a priori ed in seguito (o talvolta in contemporanea) la loro classificazione, per mezzo di un opportuno modello di classificazione.

Il compito si rivela, evidentemente, più difficile di quello della semplice classificazione di oggetti, essendo la localizzazione degli oggetti all'interno dell'immagine un problema anch'esso non banale.\\

L'obiettivo di questo lavoro di tesi è la risoluzione di una particolare istanza (\textit{task} di \textit{object detection}: si vogliono identificare, localizzare e conseguentemente ritagliare, le eventuali porzioni di un'immagine contenenti pinne dorsali di delfini. La classe di oggetti di interesse è quindi una sola.

Grazie al dominio ristretto del problema in esame, il task di rilevamento è risolto con un approccio in due fasi:
\begin{itemize}
\item dapprima, si localizzano e si ritagliano le eventuali pinne presenti nell'immagine, sfruttando una forma di conoscenza di alto livello direttamente disponibile nella rappresentazione delle immagini: il colore\footnote{L'idea di sfruttare il colore per isolare le pinne dei cetacei deriva proprio dalla differenza di tonalità tra l'acqua (tendente al blu e al verde) e le pinne (tendenti al grigio)};
\item in seguito, i ritagli sono sottoposti ad una fase di classificazione, che ne conferma la natura di 'Pinna' o ne smentisce il contenuto informativo ('No pinna').
\end{itemize}
Si rimanda direttamente al cap. \ref{esperimenti} per la descrizione degli esperimenti condotti.
\section{AlexNet}\label{alexnet}
AlexNet è una CNN creata tra il 2011 e il 2012 da Alex Krizhevsky, in collaborazione con Ilya Sutskever e Geoffrey Hinton \cite{alexnet}. La vittoria di AlexNet nella ImageNet Large Scale Visual Recognition Challenge (ILSVRC) \cite{imagenet2012}, ottenuta con un netto distacco nei confronti degli altri concorrenti, ha segnato l'inizio dell'enorme successo ottenuto dalle reti neurali profonde in svariati domini di applicazione \cite{historydl}

Il risultato principale di AlexNet, così come dichiarato dai suoi creatori nell'articolo originale, è il fatto che la profondità del modello è stato essenziale per conferirgli prestazioni così alte. L'alto costo computazionale dell'addestramento di AlexNet, reso oneroso appunto dalla profondità del modello (e quindi dal grande numero di parametri - circa 62.3 milioni) è stato affrontato con l'impiego di schede grafiche (GPU), che cominciavano in quegli anni a raggiungere notevoli potenze di calcolo.

\subsection{Architettura di AlexNet}
L'architettura di AlexNet è riportata schematicamente nella figura \ref{arc_alexnet} e con maggiore dettaglio in tabella \ref{tab_arc_alexnet}.
La rete accetta in input immagini $227\times 227$. Essa si compone di otto layer con parametri - cinque convoluzionali e tre completamente connessi. L'output dell'ultimo layer completamente connesso passa per un softmax layer a 1000 vie, il quale fornisce la distribuzione di probabilità per le 1000 classi del dataset ImageNet.

Tra ognuno degli otto strati parametrizzati sono interposti alcuni strati intermedi: ReLU layer, Local Response Normalization layer, Max Pooling layer, Dropout layer. Ognuno di questi sarà analizzato in maggiore dettaglio nei paragrafi successivi.

\begin{figure}[h]
\centering
\includegraphics[width=\textwidth ,keepaspectratio]{arc_alexnet}
\caption{Architettura originale di AlexNet \cite{alexnet}}
\label{arc_alexnet}
\end{figure}

Come si evince dalla figura \ref{arc_alexnet}, la rete è composta da due "\textit{pipeline}" parallele. Si scelse infatti di "estendere" la rete su due GPU NVIDIA\textsuperscript{\textregistered} GeForce\textsuperscript{\textregistered} GTX 580 3GB in fase di training, per raddoppiare la memoria massima disponibile (6GB in totale) per conservare la rete e i suoi parametri.

Queste GPU si prestano bene a lavorare in parallelo, poiché possono leggere e scrivere l'una sull'altra direttamente, senza passare dalla memoria della macchina host. Lo schema di parallelizzazione a due vie prevede che su ogni GPU risieda la metà dei kernel (o dei neuroni) di ciascuno strato parametrizzato. Le GPU possono comunicare tra loro solo in certi strati. In particolare, i kernel del layer convoluzionale 1 e 3 hanno in input l'intero output volume rispettivamente del layer di input e del layer convoluzionale 2, mentre i kernel dei rimanenti strati convoluzionali hanno in input la sola metà dell'output volume presente nella stessa GPU (\textit{grouped convolution}\footnote{La scelta di questo pattern di connettività fra le due GPU parallele è il risultato di un problema di cross-validation.}.\\

Sono di seguito passate in rassegna le principali scelte architetturali introdotte in AlexNet, ed alcuni dettagli relativi al suo addestramento.

\subsection*{Funzione di attivazione ReLU}
Dopo ogni strato parametrizzato, i valori delle attivazioni sono passati alla funzione attivatrice "rettificatore": $f(x)=x^{+}=\max(0,x)$ \cite{nairhinton}. Questa funzione attivatrice non-lineare e non soggetta a saturazione permette un addestramento molto più veloce delle reti convoluzionali profonde, in confronto a funzioni attivatrici fino ad allora più utilizzate come la funzione sigmoidea $f(x)=(1+\exp^{-x})^{-1}$ e la funzione tangente iperbolica $f(x)=\tanh(x)$.

\subsection*{Local Response Normalization}
È stato verificato che la seguente normalizzazione delle attivazioni, \textit{Local Response Normalization}, aumenta lievemente la capacità di generalizzazione del modello:

\[b_{x,y}^{i}=a_{x,y}^{i}/\left(k+\alpha \sum_{j=\max(0,i-n/2)}^{\min(N-1,i+n/2)}(a_{x,y}^{j})^{2}\right)^{\beta}\]

dove $a_{x,y}^{i}$ è l'attivazione del neurone ottenuto applicando il kernel $i$-esimo alla posizione $(x,y)$ e applicando in seguito la funzione ReLU, $b_{x,y}^{i}$ l'attivazione normalizzata, $N$ il numero totale di kernel del layer corrente, $k, n, \alpha, \beta$ sono iperparametri; sono stati usati i valori $k=2, n=5, \alpha=10^{-4}, \beta=0.75$.

Questa normalizzazione è adoperata solamente nel primo e nel secondo layer convoluzionale.

\subsection*{Overlapping Max Pooling}
La funzione di max pooling in AlexNet è stata caratterizzata dalla scelta di una dimensione del filtro di pooling $3\times 3$ e uno stride di $2$. È stato osservato durante la fase di training che questa funzione di \textit{max pooling con sovrapposizione} ha attenuato lievemente l'\textit{overfitting} della rete.

\subsection*{Data Augmentation}
Una delle difficoltà che si incontrano spesso quando si vuole addestrare una rete neurale con moltissimi parametri avendo a disposizione un dataset relativamente piccolo è il rischio del sovradattamento (\textit{overfitting}) della rete al training set, che compromette anche seriamente le prestazioni della rete quando le vengono presentati nuovi dati.
In AlexNet l'overfitting è stato ridotto grazie a tecniche di \textit{data augmentation}. In particolare, dopo aver ridimensionato a $256\times 256$ tutte le immagini del training set, quest'ultimo è stato "arricchito" con le seguenti immagini:
\begin{itemize}
\item Estrazione casuale di ritagli $224\times 224$ dalle immagini
\item Riflessione orizzontale ("a specchio") delle immagini
\item Somma di un'immagine e le sue componenti principali (PCA)\footnote{L'\textit{analisi delle componenti principali} (PCA, principal component analysis) è una tecnica per la semplificazione dei dati utilizzata nell'ambito della statistica multivariata. In questa sede ci limitiamo a specificare che il suo utilizzo nell'ambito della data augmentation è di evidenziare una importante proprietà delle immagini naturali, e cioè che l'identità di un oggetto è invariante rispetto ai cambi d'intensità e di colori nella sua illuminazione. Si rimanda ad esempio a \cite{PCA} per approfondimenti sulla PCA.}
\end{itemize}

\subsection*{Dropout}
Un altro modo per ridurre il problema del sovradattamento è l'impiego di tecniche di regolarizzazione dei parametri. AlexNet utilizza la tecnica del \textit{dropout} \cite{dropout}. Questa tecnica consiste nel settare a zero l'attivazione di ciascun neurone di un layer intermedio con probabilità $p$ (AlexNet impiega un dropout con $p=0.5$. I neuroni "azzerati" sono essenzialmente eliminati dalla rete e non contribuiscono né alla propagazione all'indietro del gradiente né al calcolo delle attivazioni nello strato finale (in fase di addestramento). Questa tecnica riduce il \textit{co-adattamento} tra neuroni: ogni neurone non può fare affidamento sulla presenza di altri neuroni, ed è costretto ad apprendere feature utili in congiunzione con diversi sottoinsiemi casuali degli altri neuroni, e non con un solo particolare sottoinsieme, migliorando la generalizzazione su nuovi dati.

In AlexNet, il dropout dei neuroni è utilizzato nei primi due layer completamente connessi. In fase di test, i neuroni di questi due strati sono moltiplicati per 0.5 per tenere conto dell'impiego del dropout in addestramento.

\subsection*{Addestramento di AlexNet}
Nella sua forma originale, AlexNet fu addestrato usando la discesa stocastica del gradiente con momento$=0.9$, mini-batch$=128$ e decadimento dei pesi (weight decay) $=0.0005$. Dettagli più specifici sulla fase di addestramento di AlexNet possono essere trovati nel paper originale \cite{alexnet}.

\begin{table}[h]
\caption{Architettura originale di AlexNet}
\label{tab_arc_alexnet}
\begin{tabularx}{\textwidth}{@{}llll@{}}
\toprule
N & Layer           & Attivazioni & Parametri \\ \midrule
1  &
INPUT &
$(227\times 227\times 3)$ &
\\ \midrule
2  & CONVOLUTION     & $(55\times 55\times 96)$ &\acapo{Pesi: $(11\times 11\times 3)\times 96$\\Bias: $(96)$} \\
3  & RELU            & --            & --\\
4  & NORMALIZATION   & --            & --\\
5  & MAX POOLING     & $(27\times 27\times 96)$ & -- \\ \midrule
6  & GROUPED CONVOLUTION     & $(27\times 27\times 256)$ & \acapo{Pesi: $(5\times 5\times 48)\times 128\times 2$\\Bias: $(128)\times 2$} \\
7  & RELU            & --            & --          \\
8  & NORMALIZATION   & --            & --          \\
9  & MAX POOLING     & $(13\times 13\times 256)$            &--           \\ \midrule
10 & CONVOLUTION     & $(13\times 13\times 384)$            & \acapo{Pesi: $(3\times 3\times 256)\times 384$\\Bias: $(384)$}          \\
11 & RELU            & --            &   --        \\ \midrule
12 & GROUPED CONVOLUTION     & $(13\times 13\times 384)$ & \acapo{Pesi: $(3\times 3\times 192)\times 192\times 2$\\Bias: $(192)\times 2$}  \\
13 & RELU            & --            &       --    \\ \midrule
14 & GROUPED CONVOLUTION     & $(13\times 13\times 256)$            & \acapo{Pesi: $(3\times 3\times 192)\times 128\times 2$\\Bias: $(128)\times 2$}\\
15 & RELU            & --            &   --        \\
16 & MAX POOLING     &$(6\times 6\times 256)$            &     --      \\ \midrule
17 & FULLY CONNECTED &$4096$& \acapo{Pesi: $4096\times 9216$\\Bias: $4096$} \\ 
18 & RELU            & --            &     --      \\
19 & DROPOUT         & --            &    --       \\ \midrule
20 & FULLY CONNECTED &$4096$&\acapo{Pesi: $4096\times 4096$\\Bias: $4096$}\\
21 & RELU            & --            &   --        \\
22 & DROPOUT         & --            &   --        \\ \midrule
23 & FULLY CONNECTED &$1000$&\acapo{Pesi: $2\times 4096$\\Bias: $2$}\\
24 & SOFTMAX         & --            &    --       \\
25 & CROSS-ENTROPY LOSS  & --            &   --     \\ \bottomrule                 
\end{tabularx}
\end{table}


\section{GoogLeNet}
\label{googlenet}
GoogLeNet (in origine \textit{Inception v1}) è una rete neurale convoluzionale profonda presentata nel 2014 da Christian Szegedy, ricercatore presso Google, ed il suo team di ricerca. La pubblicazione del paper originale \cite{googlenet} è avvenuta poco dopo la schiacciante vittoria riportata da GoogLeNet nella \textit{ILSVRC 2014}\footnote{\url{http://image-net.org/challenges/LSVRC/2014/}}, sia nel problema di \textit{object classification} che in quello di \textit{object detection}, introdotto nell'anno precedente.

Il grande contributo di GoogLeNet nell'ambito  del deep learning è il miglioramento nell'utilizzo delle risorse computazionali da parte di una rete profonda, grazie all'introduzione del modulo \textit{Inception} (par. \ref{inception}. In particolare, se confrontata ad AlexNet (par. \ref{alexnet}), GoogLeNet utilizza circa 10 volte meno parametri (circa 6,8 milioni) essendo però significativamente più profonda (22 layer con parametri) e performante (errore \textit{top-5} su dataset ImageNet 6,7\%).\\

GoogLeNet è stata progettata per risolvere alcuni problemi tipici delle reti convoluzionali particolarmente profonde:

\begin{itemize}

\item L'operazione di convoluzione su volumi molto profondi è computazionalmente onerosa. Per risolvere questo problema, GoogLeNet introduce l'operazione di convoluzione $1\times 1$ (par. \ref{1x1conv}).

\item Le reti neurali molto profonde sono molto sensibili al rischio di \textit{overfitting}, a causa del loro alto grado di astrazione e rappresentazione dei concetti (dovuto al grandissimo numero di parametri). Questo problema è attenuato grazie all'introduzione della già citata operazione di convoluzione $1\times 1$, all'operazione di \textit{global average pooling} (par. \ref{globalAveragePooling}) e all'usuale pratica della \textit{image augmentation} (par. \ref{googlenetAugmentation}).

\item Le parti salienti di un'immagine possono avere delle dimensioni estremamente variabili all'interno di essa. Si guardi ad esempio la figura seguente: 

\begin{figure}[h!] 
\centering
\includegraphics[width=0.9\textwidth]{cani.png}
\caption{Tre immagini di cani. L'area occupata da ciascun cane è differente e via via più piccola in ogni immagine}
\label{fig:cani}
\end{figure}

A causa di questa grande variabilità nella localizzazione dell'informazione la scelta delle dimensioni dei filtri convoluzionali diventa complessa. Un filtro più ampio è preferito quando l'informazione è distribuita su un'area vasta dell'immagine; un filtro più piccolo è adeguato in quei casi in cui l'informazione è localizzata in un'area più ristretta.\\
Questo problema è risolto con l'introduzione dei moduli \textit{inception} (par. \ref{inception}).

\item A causa della profondità di una rete, un altro tipico problema che si presenta durante l'addestramento è la scomparsa del gradiente (\textit{vanishing gradient problem}, par. \ref{vanishingGradient}), nella fase di \textit{backpropagation}. Questo problema è risolto prevedendo una particolare architettura della rete, che in fase di addestramento risulta in realtà composta da tre sottoreti diverse (par. \ref{classificatoriAusiliari}).

\end{itemize}


\subsection{Convoluzione $1\times 1$}
\label{1x1conv}
Mutuando un'idea precedentemente esposta in \cite{NiN}, GoogLeNet introduce l'operazione di convoluzione $1\times 1$ (seguita dalla funzione di attivazione ReLU). Lo scopo di questa operazione è quello di ridurre le dimensioni (in particolare, la profondità) di un volume di attivazioni per ridurre il costo computazionale delle operazioni di convoluzione successive. Questo permette di ottenere reti più profonde ma anche - con il modulo \textit{inception} (par. \ref{inception}) - più "larghe".
La fig. \ref{fig:1x1conv} mostra schematicamente l'applicazione di un'operazione di convoluzione $1\times 1$.

\begin{figure}[h] 
\centering
\includegraphics[width=0.85\textwidth]{1x1conv.png}
\caption{Convoluzione $1\times 1$ applicata ad un volume di attivazioni $6\times 6\times 32$. La profondità del volume di output dipende dal n. di filtri $1\times 1$ adoperati. In questo caso si ottiene una riduzione della profondità per $n<32$.}
\label{fig:1x1conv}
\end{figure}

Grazie all'introduzione della convoluzione $1\times 1$, inoltre, il numero di pesi nei kernel delle operazioni di convoluzione successive diminuiscono, riducendo così il rischio di overfitting.\\

A conclusione del paragrafo viene riportata un'osservazione interessante. Le reti neurali artificiali "standard" (contenenti cioè solo strati completamente connessi) possono essere convertite in reti convoluzionali mediante sole convoluzioni $1\times 1$. Infatti nelle ANN con soli strati completamente connessi tutti i volumi su cui si opera hanno dimensione $1\times 1\times k$, dove $k$ è variabile ed è il numero $k$ di neuroni dello strato che ha generato tale volume; si può allora pensare a ciascuno di questi strati completamente connessi con $k$ neuroni (biunivocamente associati al volume che generano in output) come ad uno strato di convoluzione $1\times 1$ che applica $k$ filtri al volume in input. Ogni neurone del volume di output $1\times 1\times n$ ha così una completa connettività con tutti i neuroni del volume (e quindi dello strato completamente connesso) precedente.

\subsection{Modulo \textit{Inception}}
\label{inception}
Per ovviare al problema della variabilità dell'area occupata da un oggetto da classificare in un'immagine, l'idea chiave è stata quella di operare molteplici convoluzioni, con diverse dimensioni dei kernel, in parallelo.
Così facendo la rete tende a diventare più "larga" anziché più "profonda".
Il modulo \textit{inception} è stato sviluppato proprio sulla base di questa idea.
In fig. \ref{fig:inception} è mostrato il modulo \textit{inception}, innestato più volte nell'architettura di GoogLeNet (par. \ref{architetturaGooglenet})

\begin{figure}[h] 
\centering
\includegraphics[width=\textwidth]{inception.jpg}
\caption{Modulo \textit{inception}}
\label{fig:inception}
\end{figure}

Tale modulo riceve un volume di input da uno strato precedente e opera in parallelo su di esso tre convoluzioni (con filtri $1\times 1$, $3\times 3$, $5\times 5$ rispettivamente) e un \textit{max pooling} con finestra $3\times 3$. Per diminuire il costo computazionale delle operazioni di convoluzione, si è deciso di implementare una convoluzione $1\times 1$ prima delle convoluzioni $3\times 3$ e $5\times 5$ e dopo il \textit{max pooling} $3\times 3$.
Gli output sono infine "impilati" tra loro lungo la dimensione della profondità, e il volume risultante viene inviato al successivo layer(o modulo \textit{inception}).

\subsection{Classificatori ausiliari}
\label{classificatoriAusiliari}
L'architettura di GoogLeNet prevede, per la sola fase di addestramento, delle "diramazioni", posizionate circa a metà della rete, evidenti in fig. \ref{architetturaGooglenet}.
Questi rami della rete ospitano dei classificatori ausiliari che consistono di
\begin{itemize}
\item Average Pooling $5\times 5$ (Stride 3)
\item Convoluzione $1\times 1$ (128 filtri)
\item ReLU layer
\item 1024 Fully Connected layer
\item 1000 Fully Connected layer
\item Dropout layer (70\% di probabilità di dropout di ciascun output; vd. par. \ref{dropout})
\item Softmax layer
\end{itemize}
Questa particolare architettura è stata ideata per attenuare il problema della scomparsa del gradiente (\textit{vanishing gradient problem}, par. \ref{vanishingGradient}), tipico delle reti molto profonde. L'idea si basa su un'osservazione: le promettenti performance di reti meno profonde nei task di \textit{image classification} suggeriscono che le \textit{feature} estratte dagli strati intermedi della rete hanno un grande impatto sulla predizione finale. Pertanto, nel tentativo di propagare meglio il segnale gradiente all'indietro (cioè per amplificarlo), la funzione costo calcolata è amplificata sulla base delle predizioni dei classificatori intermedi.
In fase di addestramento, dai valori di output dello strato softmax dei tre classificatori (i due intermedi e quello finale) viene calcolata la funzione costo (\textit{cross-entropy loss}). 
Il costo totale sarà costituito dalla somma pesata del costo registrato dal classificatore finale (con peso 1) e quello registrato dai due classificatori ausiliari (con peso $0.3$).

\subsection{Global Average Pooling}
\label{globalAveragePooling}
In precedenza, nella parte conclusiva delle architetture delle reti convoluzionali venivano posti due o più layer completamente connessi (ad esempio in AlexNet, par. \ref{alexnet}, che presenta tre fully connected layer finali).
Mutuando un'idea esposta in \cite{NiN}, in GoogLeNet si è previsto invece un singolo layer completamente connesso finale con 1000 neuroni (quello che fornisce valori alla funzione softmax) preceduto da uno strato di pooling denominato \textit{global average pooling}. Questo strato opera un pooling su una superficie $7\times 7$ (le dimensioni di base e altezza del volume di output precedente) e restituisce una superficie $1\times 1$, il cui valore è la media calcolata sui 49 valori dei neuroni della superficie. Gli autori hanno mostrato che passare da un fully connected layer a un average pooling layer ha contribuito a migliorare la \textit{top-1 accuracy} nella ILSVRC del 0.6\%, abbassando ulteriormente il numero dei pesi della rete (da circa 51.2 milioni se fosse stato utilizzato il FC layer agli 0 del global average pooling). In questo modo, GoogLeNet è stata resa anche leggermente più robusta all'\textit{overfitting}.
La fig. \ref{fig:FCvsGAP} mostra un confronto tra il funzionamento di un FC layer e un GAP (Global Average Pooling) layer.

\begin{figure}[h]
\centering
\includegraphics[width=0.85\textwidth]{FCvsGAP.png}
\caption{Confronto fra un FC layer e un GAP layer}
\label{fig:FCvsGAP}
\end{figure}

\subsection{Data Augmentation}
\label{googlenetAugmentation}
Per ridurre ulteriormente l'overfitting al training set di ImageNet, GoogLeNet fa uso di una particolare tecnica di \textit{data augmentation} (par. \ref{augmentation}). Al posto delle immagini originali, per l'addestramento sono stati utilizzati dei ritagli (uno per ciascuna immagine di un mini-batch) di dimensioni distribuite uniformemente tra l'8\% e il 100\% dell'intera immagine e con \textit{aspect ratio} (rapporto tra larghezza e altezza dell'immagine) distribuito uniformemente tra 3/4 e 4/3. In aggiunta, sono state operate alcune distorsioni fotometriche già adoperate in precedenza in \cite{photometric}.

\subsection{Architettura di GoogLeNet}
\label{architetturaGooglenet}
L'architettura di GoogLeNet è riportata in forma tabellare in tab. \ref{tab:tabellaGooglenet} e in forma grafica in fig. \ref{fig:googlenet}

\begin{figure}[h]
\centering
\includegraphics[width=0.9\textwidth]{tabellaGooglenet.png}
\caption{Architettura di GoogLeNet}
\label{tab:tabellaGooglenet}
\end{figure}

\begin{figure}[tb] 
\centering
\includegraphics[width=0.95\textwidth, height=0.99\textheight, keepaspectratio]{GoogLeNet.png}
\caption{Architettura di GoogLeNet}
\label{fig:googlenet}
\end{figure}

La rete accetta in input immagini $224\times 224$. I primi layer parametrizzati dall'inizio della rete sono 3 layer convoluzionali, a cui seguono 9 moduli \textit{inception} (ognuno con 6 layer convoluzionali) e infine un fully connected layer (finale).
Si evidenzia nuovamente, come già esposto nel par. \ref{classificatoriAusiliari}, che le ramificazioni che ospitano i due classificatori ausiliari esistono solo in fase di addestramento, e vengono eliminati dopo di esso.

\subsection{Addestramento di GoogLeNet}
GoogLeNet fu addestrato sul dataset di ImageNet (par. \ref{imagenet}) su una macchina che usava solamente le sue CPU per effettuare calcoli\footnote{Tuttavia gli autori affermano che se fosse stato usato un ridotto numero di GPU di fascia alta l'addestramento avrebbe potuto essere effettuato in meno di una settimana (avendo come unica limitazione la memoria delle GPU stesse).}, usando come algoritmo di ottimizzazione la discesa stocastica del gradiente con momento = 0.9, \textit{learning rate}\footnote{Per la ILSVRC 2014, il team ha in realtà utilizzato un ensemble di 7 reti GoogLeNet diverse, ognuna addestrata con \textit{learning rate} iniziale e \textit{mini-batch size} diversi} con \textit{annealing} del 4\% ogni 8 epoche. Dettagli più specifici sulla fase di addestramento di GoogLeNet possono essere trovati nel paper originale \cite{googlenet}.


%% BIBLIOGRAFIA
\backmatter

\addcontentsline{toc}{chapter}{Conclusioni}	%aggiunge la voce non numerata "Conclusioni" all'indice
\chapter{Conclusioni e sviluppi futuri}
Nel presente lavoro di tesi è stato affrontato un problema di \textit{object detection} con oggetto il rilevamento di pinne dorsali di cetacei in una collezione di immagini, adoperando tecniche di \textit{computer vision} e di \textit{deep learning}.
In particolare, si è deciso di utilizzare il metodo del \textit{transfer learning} per migliorare le prestazioni registrate dagli algoritmi CropFin v1 e v2 sviluppati in \cite{gianvito} che per primi hanno risolto il task in esame, e che costituiscono il punto di partenza degli esperimenti effettuati in questa tesi.\\

L'output degli esperimenti è stato un nuovo classificatore \textit{ensemble}, composto da tre reti neurali convoluzionali pre-addestrate di diversa profondità e capacità, adattate a risolvere il problema della classificazione binaria 'Pinna'/'No Pinna', a partire dai ritagli delle eventuali pinne di un'immagine prodotti dalla prima parte della routine CropFin v1. Questo classificatore si innesta nella routine CropFin v1 andando a sostituire il classificatore nativo, basato sull'utilizzo di una CNN \textit{from scratch}, registrando un netto miglioramento di prestazioni nella fase di classificazione di ritagli (test error 97.2\% e specificity 98.1\% su una collezione di ritagli mai vista dall'ensemble, contro il 92\% e 95\% del classificatore nativo). Il raggiungimento di tali prestazioni sono il risultato di diversi fattori: in primo luogo, l'alta capacità di rappresentazione e astrazione dei concetti raggiunte dalle tre reti utilizzate; inoltre, il dominio ristretto del problema in esame e la disponibilità di immagini per l'addestramento in numerose condizioni di scatto, che garantiscono buona generalizzazione al nostro modello di classificazione.\\

Il tentativo di un nuovo approccio al problema con la tecnica del \textit{transfer learning}, suggerito dai recenti successi registrati grazie all'utilizzo di questa tecnica (\cite{tl1}, \cite{tl2}, \cite{tl3}, tra quelli più vicini al problema in esame), è pienamente giustificato dagli ottimi risultati ottenuti in termini di prestazioni.\\

Il problema di rilevamento delle pinne di cetacei all'interno di un'immagine si inserisce in un più ampio problema di \textit{object detection}, che ha come oggetto la foto-identificazione automatica dei delfini avvistati durante le campagne di avvistamento in mare aperto. In previsione di un utilizzo dei ritagli classificati dal nostro ensemble di reti da parte di routine che compiano appunto questo foto-riconoscimento (\textit{SPIR} \cite{maglietta} e \textit{SPIR v2}\cite{emanuele}) è stato fondamentale ridurre quanto più possibile il valore di specificity (che dà una misura di quanto i falsi positivi "inquinano" i ritagli invece correttamente classificati come 'Pinna') per dare in input a queste routine solo ritagli che raffigurino effettivamente una pinna.\\

La metodologia di \textit{object detection} introdotta può essere il punto di partenza per interessanti sviluppi futuri.\\

Si può pensare di riutilizzare il classificatore mediante \textit{transfer learning} per risolvere il task di riconoscimento delle pinne dorsali anche per quelle specie marine il cui studio da parte dei biologi preveda il foto-riconoscimento degli esemplari a partire dalla loro pinna dorsale, quali orche (\textit{Orcinus orca}) \cite{orche} e squali \cite{squali}.\\

Gli algoritmi di \textit{photo-ID} dei cetacei a partire dalla loro pinna, come quello descritto in \cite{emanuele}, hanno il difetto di assegnare \emph{sempre} una "identità" probabile ad un ritaglio (prodotto ad esempio come output di CropFin v1), anche nel caso in cui esso non rappresenti davvero una pinna. In questo senso, cercare di migliorare il più possibile la specificity del classificatore su un generico test set diventa di primaria importanza. Ma non solo: anche quei ritagli che sono correttamente etichettati come pinne possono in vario modo essere non idonei al foto-riconoscimento automatizzato per mezzo di un algoritmo (ad esempio perché la pinna risulta troppo piccola, sfocata, disturbata da schizzi d'acqua: in generale sono problematici tutti quei ritagli in cui le \textit{feature} che permettono un'identificazione univoca dell'esemplare, come i noti graffi dei grampi o il bordo delle pinne dei tursiopi, sono scarsamente evidenti e pertanto inaffidabili).

Per risolvere questa criticità si può pensare di agire ulteriormente sul set dei ritagli da classificare, escludendo quelli ritenuti dalla macchina "non idonei" al successivo foto-riconoscimento, magari mediante l'uso di tecniche di \textit{computer vision} che permettano di filtrare ulteriormente i ritagli. Ad esempio, sarebbe utile riconoscere (ed escludere) quando una pinna è troppo sfocata (la macchina può verificare che il gradiente ai bordi della pinna è poco "ripido" e porta dal grigio della pinna al blu-verde dell'acqua senza significativa soluzione di continuità).
In alternativa, si può tentare un approccio che prevede il ri-addestramento del classificatore ensemble proposto in questa tesi (ancora una volta con la tecnica del \textit{transfer learning}) presentandogli come training set una collezione di ritagli divisi tra 'Idoneo' e 'Non idoneo' al foto-riconoscimento.\\



\nocite{*}
\printbibliography[heading=bibintoc]		%stampa la bibliografia alla fine e la aggiunge all'indice generale


\end{document}
