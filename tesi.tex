% !TEX encoding = UTF-8 Unicode

%% INIZIO SORGENTE
\RequirePackage{colortbl}
\documentclass[a4paper,12pt,twoside]{book}

%% pacchetti usati:
% babel, indent first, fancy headers, amssymb, amsmath, latexsym, geometry per le specifiche, parindent=1cm, frontespizio (!)

\usepackage[english,italian]{babel}
\usepackage[utf8]{inputenc}
\usepackage[T1]{fontenc}
\usepackage{indentfirst}
\usepackage{fancyhdr}
\usepackage[titletoc]{appendix} %appendici

% per far vedere le etichette, da togliere quando si deve stampare
%\usepackage{showkeys}

% per la matematica
\usepackage{amsmath}
\usepackage{latexsym}
\usepackage{amssymb}
\usepackage{mathtools}
\usepackage{stackengine}

% comandi utili
\newcommand{\eqdef}{\stackrel{\mathclap{\normalfont\mbox{def}}}{=}}
\newcommand\oast{\stackMath\mathbin{\stackinset{c}{0ex}{c}{0ex}{\ast}{\bigcirc}}}
\newcommand{\R}{\mathbb{R}}
\newcommand{\N}{\mathbb{N}}
\newcommand{\Z}{\mathbb{N}}
\newcommand{\ii}{(i)}
\newcommand{\XX}{\mathbf{X}}
\newcommand{\WW}{\mathbf{W}}
\newcommand{\bb}{\mathbf{b}}
\newcommand{\xii}{x_i}
\newcommand{\xjj}{x_j}
\newcommand{\abs}[1]{\left|#1\right|} %valore assoluto
\usepackage[top=3cm, bottom=3cm, left=3.5cm, right=2.5cm]{geometry}
\parindent=1cm
\usepackage[swapnames]{frontespizio}
\usepackage{changepage}

% per le immagini
\usepackage{graphicx}
\graphicspath{{img/}{img_esperimenti/}{img_appendice/}{img_esperimenti/foto_azzorre/}{img_esperimenti/foto_taranto/}{img_introduzione/}{img/googlenet/}{img/alexnet/}{img/resnet/}}
\usepackage{subcaption}
\usepackage{float}
\newcommand{\rulesep}{\unskip\ \vrule\ }


% per le tabelle
\usepackage{caption} %vale anche per le immagini
\captionsetup{aboveskip=10pt}
\captionsetup{belowskip=0pt}
\captionsetup[table]{position=bottom}
\usepackage{tabularx, booktabs}
\newcommand{\acapo}[1]{%
  \begin{tabular}{@{}c@{}}\strut#1\strut\end{tabular}%
}
\usepackage{adjustbox}

% per il codice sorgente
\usepackage{verbatim}

% per la bibliografia
\usepackage[autostyle, italian=guillemets]{csquotes}
\usepackage[backend=biber, style=numeric-comp, babel=hyphen, sorting=none]{biblatex}
\usepackage{url}
\usepackage{fancyvrb}
\addbibresource{biblio_tesi.bib}


%% INIZIO TESI
\frontmatter

\begin{document}

%% CREAZIONE FRONTESPIZIO
\begin{frontespizio}
\Istituzione{POLITECNICO DI BARI}
\Logo[5cm]{logo}
\Dipartimento{Ingegneria Elettrica e dell'Informazione - DEI}
\Corso[Laurea Triennale]{Ingegneria Informatica e dell'Automazione}
\Titolo{Analisi e classificazione di immagini mediante reti neurali convoluzionali per la foto-identificazione dei cetacei}
\Titoletto{Tesi di laurea in CALCOLO NUMERICO}
\Candidato[568581]{Tommaso MONOPOLI}
\NCandidato{Laureando}
\Relatore{Prof. Tiziano POLITI}
\Correlatore{Dott. Vito RENÒ}
\Annoaccademico{2018-2019}
\Rientro{1.5cm}
\Preambolo{\renewcommand{\frontlogosep}{10pt}}
\Preambolo{\renewcommand{\frontpretitlefont}{\fontsize{14}{17}\rmfamily}}
\Preambolo{\renewcommand{\fronttitlefont}{\fontsize{21}{24}\bfseries}}
\end{frontespizio}

%% SOMMARIO
%\begin{abstract}
\chapter*{Sommario}
Il sistema terrestre è sempre stato soggetto alle conseguenze delle attività umane, che mettono fortemente a rischio la biodiversità degli ecosistemi acquatici e marini. Diversi studi cercano di capire in che modo la perdita della biodiversità possa alterare l’integrità e il funzionamento di tali ecosistemi.

Una risposta a questa domanda può essere ricercata negli studi effettuati sulla distribuzione e sullo stato di conservazione dei cetacei, oggetto di numerose ricerche negli ultimi anni.
È noto, infatti, che lo stato di salute di ecosistemi con catene alimentari lunghe, come quello marino, si rifletta nella presenza e nell’abbondanza dei predatori dei livelli trofici superiori; tra questi figurano proprio i delfini.
Studiare i delfini significa, quindi, conoscere indirettamente lo stato di salute dell’ecosistema
marino in cui sono inseriti.

Una tecnica non invasiva che permette di studiare i cetacei nel loro habitat naturale è la \emph{foto-identificazione} degli esemplari, che prevede il riconoscimento - automatico o manuale - di uno stesso individuo in diverse immagini collezionate nel tempo, mediante l’analisi di suoi particolari segni distintivi (ad esempio dei graffi sul dorso, o la forma delle pinne) catturati in uno scatto.
Quest'attività può essere effettuata manualmente, ma con un grande costo in termini di tempo per i ricercatori, che spesso hanno a disposizione diverse migliaia o milioni di fotografie, scattate nel corso di anni. L'evidente difficoltà nell'approccio manuale alla foto-identificazione dei cetacei (tutt'oggi ancora ampiamente operata) suggerisce l’applicazione di metodologie di \emph{Computer Vision} e \emph{Machine Learning} per automatizzare tale attività.

L’obiettivo del presente lavoro di tesi è la creazione di un modello di classificazione binario capace di discriminare la presenza o meno di una pinna dorsale di un delfino all'interno di un'immagine. Questa è una delle operazioni che i biologi svolgono prima di poter foto-identificare gli esemplari fotografati durante le campagne di avvistamento: renderla automatica significa permettere ai biologi di risparmiare tempo e risorse, nonché indirizzare il lavoro di ricerca su task più significativi e di più alto livello.
Le metodologie impiegate sono quelle del \emph{machine learning}; in particolare, si è scelto di utilizzare la tecnica del \emph{transfer learning} per il riuso e l'adattamento di modelli pre-addestrati, usati per risolvere task di classificazione diversi da quello in esame.
Gli esperimenti condotti su dati reali acquisiti in mare permettono di rilevare l’efficacia e le alte prestazioni del classificatore oggetto di sviluppo del presente elaborato.
%\end{abstract}

\chapter*{Ringraziamenti}
Devo ringraziare innanzitutto le persone che mi hanno permesso di svolgere questo lavoro di tesi.

Ringrazio il Dott. Vito Renò, per avermi concesso l'opportunità di intraprendere questo percorso di tirocinio, seguendomi con dedizione, professionalità, continui stimoli ed una buona dose di pazienza durante tutte le fasi dello sviluppo della mia tesi di laurea e nonostante le mie innumerevoli incertezze.

Un ringraziamento al CNR per l'esperienza estremamente formativa che ho avuto il privilegio di intraprendere.

Porgo un ringraziamento al Prof. Tiziano Politi, per la sua disponibilità in qualità di relatore e per i tanti consigli dispensati.

Ringrazio tutte le persone conosciute in questi tre anni di università. In particolare, un sentito ringraziamento a Gianvito Losapio: la sua dedizione ed il rigore nello studio, la sua vivace curiosità e le innumerevoli altre qualità della sua persona sono stati per me un riferimento costante ed un inesauribile stimolo a migliorarmi.

Ringrazio la mia famiglia, per dimostrare giorno dopo giorno nei miei confronti un supporto ed un affetto sempre maggiore di quello che io dimostro loro, e che però si meritano.

Ringrazio i miei amici, scusandomi in anticipo con loro se per motivi di brevità non potrò citarli tutti. In particolare ringrazio la \emph{Corteccia} per la sincera e disinteressata amicizia che ci lega da anni. Ringrazio Marcello, per aver per primo infuso in me un po' della sua passione per lo studio e per aver condiviso con me tanti interessi in comune. Ringrazio Silvia, per i tanti bei momenti musicali che abbiamo passato e che passeremo insieme. Ringrazio Cristina, a cui se davvero potessi dedicherei 15 pagine di ringraziamenti (come tante volte le ho promesso).


%% INDICE GENERALE
\tableofcontents
%\listoffigures
%\listoftables

%%
%% --- MATERIALE PRINCIPALE ---
%%
\mainmatter

%% 1 - INTRODUZIONE
\chapter{Introduzione}
\label{introduzione}
%\addcontentsline{toc}{chapter}{Introduzione}
La foto-identificazione è una tecnica largamente impiegata per l’identificazione dei singoli individui a partire da una o più immagini. Il principale vantaggio di questa tecnica è la sua non invasività che la rende particolarmente utile per studiare sia la dinamicità che i movimenti di ogni specie. Tale metodologia risulta essere uno strumento affidabile quando viene applicato nella comprensione dei comportamenti dei cetacei (migrazioni e spostamenti). Tra i delfini, vi sono due specie più adatte a tali studi.  Il primo delfino riguarda la specie “Tursiops truncatus” (tursiope) o delfino dal naso a bottiglia, mentre la seconda specie riguarda la specie “Grampus griseus” (Grampo, o delfino di Risso), avente numerose cicatrici su tutto il corpo, entrambi appartenenti alla famiglia dei Delfinidi.[1]

\section{Problema e obiettivi}
L'obiettivo principale del presente lavoro di tesi è stato quello di creare un sistema per il rilevamento automatico della presenza di cetacei all'interno di uno scatto fotografico.

Lo scopo finale è facilitare lo studio dei cetacei, attorno al quale si riunisce grande interesse scientifico (par. \ref{TODO}), incentivando l'utilizzo di tecniche non invasive basate su algoritmi innovativi e grandi disponibilità di dati. Uno dei principali metodi di studio non-invasivi dei cetacei è la foto-identificazione degli esemplari (par. \ref{fotoidentificazione}). Il presente lavoro di tesi rappresenta un passo avanti verso la completa automatizzazione del processo di foto-identificazione degli esemplari incontrati durante le campagne di avvistamento, fornendo un miglioramento nelle prestazioni di una routine (CropFin v1, par. \ref{cropFin}) che può aiutare i biologi nel successivo lavoro di foto-identificazione.

Il problema affrontato TODO

%% 2 - METODOLOGIE
\chapter{Metodologie}\label{teoria}
\pagestyle{fancy}
\fancyhf{}
\fancyhead[OL]{\rightmark}
\cfoot{\thepage}

Nel corso di questo capitolo saranno introdotti progressivamente i concetti principali su cui si fonda la \textit{computer vision} (visione artificiale) e il \textit{machine learning} (apprendimento automatico). Questi presupposti teorici permetteranno di comprendere da un punto di vista teorico quanto descritto nella sezione sperimentale della tesi (cap. \ref{esperimenti}.

Per i primi paragrafi, dedicati all'\textit{image processing}, le principali fonti seguite sono \cite{gianvito} e \cite{computerVision}.

Per i paragrafi dedicati al \textit{machine} e \textit{deep learning} le fonti di riferimento sono \cite{dlbook} e soprattutto \cite{cs231n}.


\section{Immagini digitali}
Caratterizziamo intuitivamente il concetto di "immagine" dal punto di vista informatico.

Un'\textbf{immagine digitale} è una rappresentazione binaria di un'immagine (in generale a colori) a due dimensioni\footnote{Ci riferiamo in questa sede solo alle immagini di tipo raster, quelle cioè con risoluzione e numero di canali di colore fissati a priori, come ad esempio le immagini digitali in formato jpg.}; essa può essere definita matematicamente come un tensore $\mathcal{I}\in\{0,\dots,255\}^{h\times w\times c}$, dove $h$ e $w$ sono rispettivamente dette \textbf{altezza} e \textbf{larghezza} dell'immagine, la coppia $(w,h)$ \textbf{risoluzione} mentre $c$ è il numero di \emph{canali di colore}\footnote{Spesso si scrive che $\mathcal{I}$ è un immagine $w\times h\times c$, o più semplicemente $w\times h$ (assumendo $c=3$)}. Nello \textbf{spazio di colore sRGB} (standard RGB, d'ora in avanti abbreviato in RGB), ampiamente adoperato, i canali di colore sono rosso (R, Red), verde (G, Green) e blu (B, Blue), quindi $c=3$. In mancanza di diverse indicazioni, ci si riferirà nel seguito allo spazio di colore RGB.

Un \textbf{pixel} $p(i,j)$ è definito come la funzione vettoriale
\[p(i,j)=[r(i,j),g(i,j),b(i,j)]\]
essendo $r,g,b:\{0,\dots,h\}\times\{0,\dots,w\}\to\{0,\dots,255\}$ le funzioni scalari che associano ad ogni posizione bidimensionale $i,j$ dell'immagine un valore intero di \textbf{intensità luminosa} compreso tra 0 e 255, uno per ciascuno dei tre canali RGB.
Ogni pixel definisce univocamente un colore nello spazio RGB, il quale può quindi rappresentare in tutto $256^{3}$ colori diversi, cioè circa 17 milioni.

Si può immaginare il tensore immagine $\mathcal{I}$ come una "pila" di tre matrici, una per ogni canale di colore, come mostrato in figura \ref{fig:rappresentazione_tensore}.

\begin{figure}[h]
\centering
\includegraphics[scale=0.7]{rappresentazione_tensore}
\caption{Rappresentazione grafica di un tensore tridimensionale; in ogni posizione compaiono gli indici del tensore}
\label{fig:rappresentazione_tensore}
\end{figure}

Un esempio di immagine digitale, scomposta nelle sue componenti RGB, è mostrato in figura \ref{fig:canali_rgb}.

\begin{figure}[h]
  \begin{minipage}[b]{0.46\textwidth}
    \includegraphics[width=\textwidth]{canali_rgb}
    \caption{Canali RGB di un'immagine}
    \label{fig:canali_rgb}
  \end{minipage}
  \hfill
  \begin{minipage}[b]{0.46\textwidth}
    \includegraphics[width=\textwidth]{pixel}
    \caption{Pixel di un'immagine}
  \end{minipage}
\end{figure}

In Matlab un'immagine digitale può essere rappresentata con il tipo di dato \verb|multidimensional-array| con tre dimensioni (corrispondenti ai tre canali di colore RGB), in cui ciascun elemento è di tipo \verb|uint8| (ma può appartenere anche ad altri tipi di dato\footnote{\url{https://it.mathworks.com/help/matlab/ref/image.html?s_tid=doc_ta#buqdlnb-C}}).

È possibile importare un'immagine RGB con la funzione:
\begin{verbatim}
im = imread("immagine.jpg");
\end{verbatim}
I canali Rosso \verb|R|, Verde \verb|G| e blu \verb|B| possono essere ottenuti come:
\begin{verbatim}
R = im(:,:,1);
G = im(:,:,2);
B = im(:,:,3);
\end{verbatim}
Il pixel di posizione $(i,j)$ è quindi:
\begin{verbatim}
p = [im(i,j,1) im(i,j,2) im(i,j,3)];
\end{verbatim}

\subsection{Spazio di colore Lab}
Oltre al modello RGB descritto al paragrafo precedente, esistono ulteriori spazi di colore che consentono di ottenere una differente rappresentazione delle medesime tonalità dei pixel.
Nel presente lavoro di tesi (par. \ref{faseRitaglio}) si utilizza una conversione delle immagini dallo spazio di colore iniziale RGB allo spazio di colore \textbf{CIE 1976 L*a*b*} (nel seguito abbreviato come Lab).
Nello spazio Lab le tonalità sono ancora espresse da triplette di valori (L*, a* e b*), ma con un significato diverso rispetto a RGB: il valore L* rappresenta la luminanza
(variazione di luminosità), mentre le altre due rappresentano la crominanza (variazione
di colore), rispettivamente una scala verde-rosso (a*) ed una scala blu-giallo (b*).

A partire dall’immagine \verb|im| è possibile convertire lo spazio di colori da RGB a Lab, ottenendo una nuova immagine \verb|im_lab|:
\begin{verbatim}
im_lab = rgb2lab(im);
\end{verbatim}
I dettagli sulla conversione di spazio di colore possono essere letti al par. 2.1.1 di \cite{gianvito}.

\subsection{Collezioni di immagini in Matlab}
Una delle necessità principali del lavoro affrontato è stata la gestione di grandi quantità di immagini. Si riportano i tipi di dato messi a dispozione da Matlab e utilizzati negli esperimenti del presente lavoro di tesi:

\begin{itemize}

\item \verb|imageDatastore|: oggetto progettato appositamente per gestire ed elaborare rapidamente una grande quantità di immagini. Per istanziare un oggetto \verb|imageDatastore| bisogna specificare l'argomento \verb|path| che indica il percorso della collezione di immagini da importare. Altri argomenti (coppie argomento-valore) opzionali per l'inizializzazione di questo oggetto sono:
\begin{itemize}
\item \verb|’IncludeSubfolders’,true|\\
Include le immagini contenute nelle sottocartelle di \verb|path| 
\item \verb|’LabelSource’,’foldernames’|\\
Assegna a ciascuna immagine un’etichetta
data dal nome della cartella in cui è contenuta
\end{itemize}
Quindi, per creare l'oggetto:
\begin{verbatim}
imds = imageDatastore(path,’IncludeSubfolders’,true,...
’LabelSource’,’foldernames’);
\end{verbatim}
L’elenco delle immagini è restituito nel campo \verb|imds.Files|.

\item \verb|augmentedImageDatastore|: oggetto creato a partire da un \verb|imageDatastore| applicando operazioni di preprocessing specificate in un oggetto \verb|imageDataAugmenter|. La sintassi di questo oggetto verrà approfondita nel par. \ref{augmentation}. TODO

\end{itemize}

\section{Trasformazioni}
Si riportano le operazioni fondamentali che hanno consentito, nel presente lavoro di tesi, di trasformare le immagini in una forma maggiormente adatta ad un’analisi successiva, una strategia chiamata \textit{image augmentation} (par. \ref{augmentation}).

\subsection{Ridimensionamento}
Il ridimensionamento consente, in generale, di ottenere una nuova immagine a risoluzione differente, riducendo o aumentando il numero di pixel utilizzati per rappresentarla.
Nel caso del presente lavoro, le dimensioni delle immagini sono state ridotte (par. \ref{faseRitaglio}), per diminuire il costo computazionale delle operazioni successive.
Per ridimensionare un'immagine \verb|im| in Matlab si può usare la funzione \verb|imresize|\footnote{La funzione imresize attua, per lo scopo, una tecnica avanzata di calcolo numerico: l’interpolazione bicubica. (TODO ? par. 2.2.1 di \cite{gianvito} per i dettagli.)}, specificando la nuova lunghezza \verb|w'| e la nuova altezza \verb|h'|:
\begin{verbatim}
im_res = imresize(im,[h' w']);
\end{verbatim}

\subsection{Rotazione}
La rotazione di un'immagine digitale può essere effettuata calcolando per ogni suo pixel $(x,y)$ il prodotto matriciale
\begin{equation*}
\begin{bmatrix}
x' \\
y'
\end{bmatrix} = 
\begin{bmatrix}
\cos\alpha & -\sin\alpha \\
\sin\alpha & \cos\alpha
\end{bmatrix}
\begin{bmatrix}
x \\
y
\end{bmatrix}
\end{equation*}
in cui $\alpha$ è l’angolo di rotazione misurato in senso antiorario rispetto all’asse x e $(x',y')$ le coordinate del pixel trasformato. L'immagine ruotata è l'insieme dei pixel $(x',y')$ calcolati.

In Matlab, la rotazione di un'immagine \verb|im| di un angolo \verb|a| può essere effettuata con
\begin{verbatim}
rotated = imrotate(im, a);
\end{verbatim}

\subsection{Riflessione}
La riflessione rispetto all’asse x di un'immagine digitale può essere effettuata calcolando per ogni suo pixel $(x,y)$ il prodotto matriciale
\begin{equation*}
\begin{bmatrix}
x' \\
y'
\end{bmatrix} = 
\begin{bmatrix}
1 & 0 \\
0 & -1
\end{bmatrix}
\begin{bmatrix}
x \\
y
\end{bmatrix}
\end{equation*}
dove $(x',y')$ sono le coordinate del pixel trasformato. L'immagine riflessa è l'insieme dei pixel $(x',y')$ calcolati.

In Matlab, la riflessione di un'immagine \verb|im| rispetto all'asse x può essere effettuata con
\begin{verbatim}
flipped = flipdim(im, 2);
\end{verbatim}

\subsection{Traslazione}
La traslazione orizzontale, a differenza delle precedenti operazioni, è una trasformazione affine ma non lineare, quindi
la rappresentazione con le matrici richiede il passaggio ad un diverso tipo di coordinate
dette omogenee:
\begin{equation*}
\begin{bmatrix}
x&y
\end{bmatrix}^\top \mapsto
\begin{bmatrix}
x&y&1
\end{bmatrix}^\top
\end{equation*}
A questo punto, è possibile ottenere le nuove coordinate traslate $(x', y')$ calcolando:

\begin{equation*}
\begin{bmatrix}
x'\\y'\\1
\end{bmatrix} = 
\begin{bmatrix}
1 & 0 & t_x \\0 & 1 & t_y \\0 & 0 & 1
\end{bmatrix}
\begin{bmatrix}
x \\ y \\ 1
\end{bmatrix}
\end{equation*}

In Matlab, la traslazione orizzontale di \verb|t| pixel di un'immagine \verb|im| può essere effettuata con
\begin{verbatim}
translated = imtranslate(im,[t 0]);
\end{verbatim}
dove \verb|t|>0 implica una traslazione verso destra, \verb|t|<0 verso sinistra.\\

I risultati delle quattro trasformazioni finora viste sono visualizzate in figura \ref{fig:trasformazioni}

\begin{figure}[h]
\centering

\begin{subfigure}[b]{\textwidth}
\centering
\includegraphics[width=0.24\textwidth]{pinnaOriginale}
\caption{Immagine originale}
\end{subfigure}

\begin{subfigure}[b]{0.24\textwidth}
\centering
\includegraphics[width=0.4\textwidth]{pinnaOriginale}
\caption{Ridimensionata}
\end{subfigure}
\begin{subfigure}[b]{0.24\textwidth}
\centering
\includegraphics[width=\textwidth]{pinnaRuotata}
\caption{Ruotata di 20$^\circ$}
\end{subfigure}
\begin{subfigure}[b]{0.24\textwidth}
\centering
\includegraphics[width=\textwidth]{pinnaRiflessa}
\caption{Riflessa}
\end{subfigure}
\begin{subfigure}[b]{0.24\textwidth}
\centering
\includegraphics[width=\textwidth]{pinnaTraslata}
\caption{Traslata di -100px}
\end{subfigure}

\caption{Trasformazioni applicate ad un'immagine digitale}
\label{fig:trasformazioni}
\end{figure}

\subsection{Convoluzione}
Nell’ambito dell’\textit{image processing} esiste una quantità notevole di operatori definiti "locali", che operano cioè non su un singolo pixel ma su un gruppo di pixel contigui.
L’operatore locale maggiormente usato è l’operatore di convoluzione. Nell'ambito del presente lavoro, esso ha un'importanza centrale per la \textit{feature extraction} delle immagini, all'interno delle reti neurali convoluzionali (par. \ref{CNN}), e più specificamente nei layer convoluzionali (par. \ref{convLayer}).


Nel seguito si presenta, pertanto, una definizione rigorosa dell'operazione di convoluzione e si forniscono semplici esempi di implementazione.

Dati due segnali discreti $x(k)$ e $w(k)$, si definisce \textbf{(somma di) convoluzione tra $x$ e $w$} una nuova funzione $s(k)$ definita come \begin{equation*}
s(k)=x(k)\ast w(k)\eqdef \sum_{i=-\infty}^{+\infty}x(i)w(k-i), i\in\Z
\end{equation*}

Si dimostra che questa operazione gode della proprietà commutativa, associativa e distributiva rispetto alla somma.

Nell'ambito della \textit{computer vision} si utilizza una versione multidimensionale dell'operazione di convoluzione, utilizzando la seguente terminologia:
\begin{itemize}
\item il primo segnale $x$ è detto \textbf{input}, generalmente costituito da un'immagine od una sua elaborazione;
\item il secondo segnale $w$ è detto \textbf{filtro} o \textbf{kernel}, solitamente costituito da una matrice
di dimensioni ridotte rispetto all'input;
\item il risultato $s$ è detto \textbf{feature map}, poiché l'operazione di filtraggio è spesso utilizzata per l'estrazione di feature a partire dall'input (par. \ref{TODO}).
\end{itemize}
È ovvio che la somma di infiniti termini prevista dalla definizione di convoluzione con segnali discreti si riduce ad una somma limitata alle dimensioni dell'immagine. Nel caso in cui l'input sia una matrice bidimensionale, ad esempio un'immagine in scala di grigi $I\in\R^{h\times w}$, anche il kernel impiegato è solitamente una matrice bidimensionale di dimensioni ridotte $K\in\R^{a\times b}$, ottenendo l’operazione:
\begin{equation*}
s(i,j)=(I\ast K)(i,j)=\sum_{k=1}^a\sum_{r=1}^b I(i+k-1,j+r-1)K(l-k+1,w-r+1)
\end{equation*}

Molte librerie, in realtà, implementano la convoluzione attraverso la funzione di crosscorrelazione (indicata con $\oast$).
In tal caso ciascun elemento della feature map si ottiene come
\begin{equation*}
s(i,j)=(I\oast K)(i,j)=\sum_{k=1}^a\sum_{r=1}^b\mathcal{I}(i+k-1,j+r-1)K(k,r)
\end{equation*}
L'operazione appena esposta - che si dimostra essere equivalente ad una convoluzione tra $I$ e $K$ - ha una semplice interpretazione, visualizzata in fig. \ref{fig:interpretazioneConvoluzione}. Convolvere un'immagine con un kernel equivale a far scorrere la matrice che rappresenta il kernel lungo l'immagine, e sviluppare i prodotti \textit{element-wise} (termine a termine) e sommarli tra loro, ottenendo l'elemento della feature map.

\begin{figure}[h]
\centering
\includegraphics[width=0.5\textwidth]{interpretazioneConvoluzione}
\caption{Un esempio di cross-correlazione (ovvero convoluzione discreta 2-D senza il
ribaltamento del kernel)}
\label{fig:interpretazioneConvoluzione}
\end{figure}

Nel caso (più comune) in cui l’input sia un tensore (ad esempio un’immagine a colori) $\mathcal{I}\in\R^{h\times w\times c}$ si richiede che il kernel abbia lo stesso numero $c$ di livelli, ad esempio $\mathcal{K}\in\R^{a\times b\times c}$. L'operazione viene così ridefinita.
\begin{equation*}
s(i,j)=(\mathcal{I}\oast\mathcal{K})(i,j,k)=\sum_{k=1}^a\sum_{r=1}^b\sum_{s=1}^c\mathcal{I}(i+s-1,j+r-1,k)K(r,s,k)
\end{equation*}
Come si nota, la feature map ottenuta sarà sempre bidimensionale (a prescindere dalla profondità $c$ del tensore in input).

Completeremo la trattazione sulla convoluzione nel par. \ref{convLayer} sul layer convoluzionale di una CNN.

\section{Problemi di \textit{Computer Vision}}
In questa sezione si riportano alcuni problemi caratteristici della \textit{computer vision}, affrontati nel corso del lavoro e finalizzati ad una comprensione di alto livello del contenuto delle immagini (e dei video) da parte del computer. Il nome italiano della disciplina, "visione artificiale", richiama in questo senso l'obiettivo di rendere artificiali i compiti svolti dal sistema visivo umano.

TODO TODO TODO vd 2.3 gianvito

\subsection{Segmentazione}
TODO
\paragraph*{Soglia di Otsu}
TODO
\paragraph*{Regioni connesse}
TODO
\paragraph*{Riempimento degli \textit{holes}}
TODO

\subsection{Object recognition}
\label{objectRecognition}
L'\textbf{\textit{object recognition}} (in italiano: riconoscimento di oggetti) nell'ambito della visione artificiale è il problema di assegnare una descrizione testuale o una o più etichette ad un'immagine, tipicamente sulla base di uno o più determinati oggetti che un computer riesce a riconoscere all'interno di essa.

Ogni categoria esistente di oggetti ha delle caratteristiche fondamentali che la differenziano da qualunque altra categoria di oggetti. Attraverso tecniche di \textit{machine learning} è possibile ricavare una descrizione di una certa categoria addestrando la macchina a riconoscere le caratteristiche (\textit{features}) fondamentali di quella categoria di oggetti, a partire da un insieme di immagini campione afferenti a quella categoria.

Per rendere affidabile il riconoscimento, è importante che l'insieme di caratteristiche estratte da ogni immagine campione sia insensibile a variazioni del punto di vista, di scala, delle condizioni di illuminazione, alle distorsioni geometriche, all'occlusione dell'oggetto, al \textit{clutter} (ingombro) di altri oggetti non informativi sullo sfondo, alle variazioni intra-classe dell'oggetto, come mostrato in fig. \ref{fig:variabilita}.

\begin{figure}[h]
\centering
\includegraphics[width=0.9\textwidth]{variabilita}
\caption{I principali "ostacoli" al riconoscimento automatico degli oggetti} 
\label{fig:variabilita}
\end{figure}

L'uomo riconosce una moltitudine di oggetti in immagini con poco sforzo, nonostante i fattori di variabilità descritti. Questo compito è ancora una sfida aperta per la computer vision in generale.\\

Il problema dell'\textbf{\textit{image classification}} è uno specifico problema di \textit{object recognition} che consiste nell'assegnare una singola etichetta (o una distribuzione di probabilità su più etichette) ad un'immagine da un insieme fisso di etichette (anche dette "categorie" o "classi"). In fig. \ref{fig:imageClassification} è visualizzato il problema in esame.

\begin{figure}[h]
\centering
\includegraphics[width=0.7\textwidth]{imageClassification}
\caption{Il task della \textit{image classification} consiste in questo caso nel calcolare una distribuzione di probabilità su quattro etichette (gatto, cane, cappello, tazza) per un'immagine digitale.} 
\label{fig:imageClassification}
\end{figure}

In letteratura sono stati proposti numerosi metodi per la risoluzione efficiente dei task di \textit{object recognition}, attraverso l'impiego di diverse tecniche di \textit{machine learning} (ad esempio l'algoritmo di clustering \textit{k-Nearest Neighbor}). Negli ultimi anni i risultati più promettenti sono stati offerti dalle \textit{reti neurali convoluzionali}, di cui parleremo diffusamente nel seguito del capitolo (par. \ref{CNN}) e che verranno impiegate negli esperimenti (par. \ref{faseClassificazione}).

\section{Object detection}
Un altro importante problema di \textit{computer vision} è l'\textbf{\textit{object detection}} (in italiano: rilevamento di oggetti). Esso consiste nella localizzazione di oggetti di categorie stabilite a priori ed in seguito (o talvolta in contemporanea) la loro classificazione, per mezzo di un opportuno modello di classificazione.

Il compito si rivela, evidentemente, più difficile di quello della semplice classificazione di oggetti, essendo la localizzazione degli oggetti all'interno dell'immagine un problema anch'esso non banale.\\

L'obiettivo di questo lavoro di tesi è la risoluzione di una particolare istanza (\textit{task} di \textit{object detection}: si vogliono identificare, localizzare e conseguentemente ritagliare, le eventuali porzioni di un'immagine contenenti pinne dorsali di delfini. La classe di oggetti di interesse è quindi una sola.

Grazie al dominio ristretto del problema in esame, il task di rilevamento è risolto con un approccio in due fasi:
\begin{itemize}
\item dapprima, si localizzano e si ritagliano le eventuali pinne presenti nell'immagine, sfruttando una forma di conoscenza di alto livello direttamente disponibile nella rappresentazione delle immagini: il colore\footnote{L'idea di sfruttare il colore per isolare le pinne dei cetacei deriva proprio dalla differenza di tonalità tra l'acqua (tendente al blu e al verde) e le pinne (tendenti al grigio)};
\item in seguito, i ritagli sono sottoposti ad una fase di classificazione, che ne conferma la natura di 'Pinna' o ne smentisce il contenuto informativo ('No pinna').
\end{itemize}
Si rimanda direttamente al cap. \ref{esperimenti} per la descrizione degli esperimenti condotti.
\section{Machine Learning e Deep Learning}
\label{machineLearning}
Il \textbf{\textit{machine learning}} è una branca dell'intelligenza artificiale (AI) che si occupa in generale di fornire ad una macchina la capacità di apprendere automaticamente dall'esperienza, essendo esplicitamente programmata su "come imparare" ma non su "cosa imparare".

Questa capacità è fornita da un certo algoritmo di machine learning, di cui T. Mitchell fornisce una celebre ed elegante definizione:
\begin{quote}
Si dice che un programma per computer impara dall’esperienza E rispetto ad una qualche classe di compiti T ed una misura di performance P se le sue prestazioni nei compiti in T, misurate attraverso P, migliorano con l’esperienza E.
\end{quote}

\noindent L'obiettivo principale dell'apprendimento automatico è che una macchina sia in grado di generalizzare dalla propria esperienza\cite{bishop}, ossia che sia in grado di svolgere ragionamenti induttivi. In questo contesto, per generalizzazione si intende l'abilità di una macchina di portare a termine in maniera accurata esempi o compiti nuovi, che non ha mai affrontato, dopo aver fatto esperienza su un insieme di dati di apprendimento.
Gli esempi di addestramento (in inglese chiamati \textit{training examples}) si assume provengano da una qualche distribuzione di probabilità, generalmente sconosciuta e considerata rappresentativa dello spazio delle occorrenze del fenomeno da apprendere; la macchina ha il compito di costruire un modello probabilistico generale dello spazio delle occorrenze, in maniera tale da essere in grado di produrre previsioni sufficientemente accurate quando sottoposta a nuovi casi.

Quasi tutti gli algoritmi di machine learning sono costruiti combinando almeno quattro
building blocks fondamentali: un dataset, un modello, una funzione costo ed un
metodo di ottimizzazione.\\

Il \textbf{\textit{deep learning}} è un tipo specifico di machine learning, che negli ultimi anni ha dato una svolta decisiva alla più vasta branca dell’intelligenza artificiale.

Un algoritmo di deep learning si basa su diversi livelli di rappresentazione dei dati, che corrispondono a differenti livelli di astrazione; questi livelli formano una gerarchia di concetti, in cui i concetti di più alto livello sono definiti sulla base di quelli di livello più basso.

Il modello computazionale utilizzato in maniera esclusiva nel \textit{deep learning} è la rete neurale artificiale (par. \ref{ANN})\\

Il deep learning è inquadrato in una più ampia branca del machine learning, chiamata \textbf{\textit{representation learning}}. Nell'ambito del representation learning, l'approccio usato per la risoluzione dei problemi consiste non solo nel tentare di addestrare un computer a trovare una relazione tra dati e output, ma anche nel fornirgli strumenti per la rappresentazione stessa dei dati, in quanto in alcuni contesti il legame tra i dati in input e quelli attesi in output è tutt'altro che lineare ed è in generale complesso, e può essere più facile per la macchina acquisire diversi livelli di conoscenza e di astrazione dei dati in input per calcolare l'output.

Si pensi ad esempio ad un task di \textit{image classification}. È impensabile descrivere la presenza di un oggetto all’interno di un'immagine attraverso un legame lineare con i singoli pixel. È piuttosto la combinazione di pixel l'informazione da mappare (in maniera in generale non lineare) con la categoria di appartenenza di quell'oggetto.

Dagli studi scientifici sull'apparato visivo sappiamo che l'uomo riesce a riconoscere gli oggetti attraverso una rappresentazione gerarchica degli stessi:

\begin{itemize}
\item dapprima captiamo caratteristiche di basso livello degli oggetti che vediamo, come forme (spigoli, angoli) e tonalità di colore. Tutte queste caratteristiche sono "locali", nel senso che occupano una regione limitata del campo visivo, essendo parti più piccole dell'oggetto. In questa fase, non sappiamo ancora dare un nome all'oggetto su cui ci concentriamo;
\item queste caratteristiche (\textit{features}) sono combinate nella parte del cervello che si occupa della visione, a formare concetti di più alto livello come il perimetro dell'oggetto e le sfumature (gradienti) di colore;
\item questa rappresentazione graduale e gerarchica dei concetti, arrivata a concetti di più alto livello, permette infine di associare alla "immagine" formatasi nel nostro campo visivo il nome dell'oggetto, o degli oggetti, in esso presenti.
\end{itemize}

Sulle medesime basi "biologiche" si fondano le reti neurali artificiali, attraverso gli \textit{hidden layers} (strati nascosti) frapposti tra l'input e l'output della rete stessa che permettono di combinare le informazioni provenienti dagli strati precedenti per ottenere una rappresentazione dei dati di più alto livello.

\begin{figure}[h]
\centering
\includegraphics[width=0.7\textwidth]{diagrammaIA}
\caption{Un diagramma di Venn che mostra come il deep learning sia un tipo di
representation learning, che a sua volta è un tipo di machine learning.}
\label{diagrammaAI}
\end{figure}


\subsection{Supervised Learning}
\label{supervisedLearning}

Il paradigma dell'\textbf{apprendimento supervisionato} (\textit{supervised learning}) mira alla creazione di un algoritmo che analizzi dei "dati di addestramento", una collezione di esempi ideali costituiti da coppie di input e relativi output attesi, e da questi inferisca una funzione che può essere usata per mappare nuovi input ai corretti output \cite{Russell2009}.
Ciò richiede all'algoritmo la capacità di trovare una funzione che sappia generalizzare efficacemente dai dati di training, al fine di adattarsi bene a nuovi dati (per poterne mappare correttamente quanti più possibile).\\

Molti problemi pratici, come ad esempio la regressione e la classificazione delle immagini (par. \ref{objectRecognition}, possono essere formulati ricorrendo ad una funzione matematica
\[\mathcal{F}:X\to Y\]
che associa ad ogni elemento dello spazio degli input $X$ uno ed un solo elemento dello spazio degli output $Y$.
Il concetto di funzione implica l'esistenza di un solo elemento di $Y$ a cui ogni elemento di $X$ è correttamente associato. Il problema consiste allora nel cercare una funzione $\mathcal{F}$ in grado di ottenere esattamente tale associazione, per quanti più elementi di $X$ possibile.

È evidente che questo tipo di problemi ben si presta ad essere approcciato con algoritmi di apprendimento supervisionato.

Prima di analizzare in dettaglio il problema di classificazione delle immagini oggetto della presente tesi, è necessario inquadrare il problema partendo da alcune definizioni preliminari.\\

Un \textbf{dataset} X è una generica collezione di $N$ dati
\[X=\{\mathbf{x}^{(1)},\mathbf{x}^{(2)},\dots,\mathbf{x}^{(N)}\}\]
Ogni dato $\mathbf{x}^{(i)}$ è chiamato \textbf{esempio} (o \textbf{data point}).
I data point possono essere anche non omogenei tra loro (cioè avere dimensioni differenti).
Ciascun esempio si può caratterizzare come un vettore $\mathbf{x}^{(i)}\in\R^{D}$, in cui ciascun elemento $x_i$ è detto \textbf{feature} e rappresenta una caratteristica di un oggetto o un evento misurato. $D$ è il numero di feature in ogni esempio, o \textbf{dimensione} dell'esempio.
In caso di esempi omogenei (cioè aventi stessa dimensione $D$) un dataset può essere descritto attraverso una matrice detta \textbf{design matrix}, in cui ogni riga corrisponde ad un particolare esempio e ogni colonna corrisponde ad una precisa feature.
Un dataset di cardinalità $N$ e in cui ogni esempio ha $D$ feature ha quindi una design matrix di dimensione $N\times D$. \\

In un problema di classificazione delle immagini sussiste la seguente caratterizzazione:
\begin{itemize}
\item $X$: un insieme di $N$ immagini digitali
\item $Y$: un insieme di $K$ classi predefinite di oggetti che possono essere individuati all'interno di un'immagine (possono essere dei "descrittori" testuali o, equivalentemente, dei numeri interi)
\end{itemize}
Un elemento di $Y$ è solitamente chiamato \textbf{etichetta} o \textbf{categoria} (in inglese \textbf{label} o \textbf{class}); si dice quindi che ogni immagine $\mathbf{x}^{(i)}\in X$ può essere \textit{descritta da un'etichetta} (o \textit{associata ad una categoria}) $\mathbf{y}^{(i)}\in Y$ tramite una funzione di associazione $f$.\footnote{Teoricamente una stessa immagine potrebbe essere descritta da più di un'etichetta o addirittura da nessuna, coerentemente col fatto che in essa potrebbero essere presenti più oggetti o nessun oggetto tra quelli previsti in $Y$. Tuttavia nella presente tesi questa ambiguità non può sussistere: la classificazione riduce qualsiasi immagine ad una di due categorie mutualmente esclusive e di cui soltanto una è quella corretta, cioè la presenza o meno di una pinna nell'immagine.}\\

Tipicamente, per l'addestramento di un modello di machine learning si usa un sottoinsieme del dataset $X$ a disposizione. Questo sottoinsieme $X^{train}$ è definito \textbf{\textit{training set}}.

Nella pratica, l'espressione analitica di $f$ non può essere trovata esattamente. Ad esempio, nel problema in esame, non è chiaro come poter scrivere un algoritmo che consenta di individuare con esattezza la presenza di una pinna all'interno di un'immagine, poiché il concetto di "pinna" non è un concetto matematico e non può essere reso facilmente in un linguaggio "comprensibile" da un computer. Inoltre, il computer può disporre solamente di un numero limitato $\left|X^{(train)}\right|$ di esempi, cioè un numero limitato di occorrenze di oggetti di tipo "pinna", che non permettono di generalizzare a tutte le forme ed i colori in cui una pinna può essere presente in un'immagine. Per questo motivo, ciò che il modello di machine learning da addestrare è chiamato ad imparare è un'approssimazione quanto più "plausibile" di $f$, dove per "plausibilità" si intende la capacità della funzione trovata di restituire il giusto output per il maggior numero possibile di input presentati.

\subsection{Underfitting e overfitting}
\label{overfitting}
La sfida principale dell'apprendimento automatico è quella di rendere l'algoritmo di machine learning performante su nuovi input, diversi da quelli su cui il modello è stato addestrato. Questa abilità è chiamata \textbf{generalizzazione}.\\

Tipicamente, dopo l'addestramento di un modello di machine learning con un \textit{training set} possiamo misurare le performance del modello con un parametro detto \textit{training error}, definito come il rapporto tra il numero di esempi del training set che alla fine dell'addestramento l'algoritmo associa al giusto output, $\left|X_{corrette}\right|<\left|X\right|$, e la dimensione del \textit{training set}, $\left|X\right|$:
\[\text{Training Error}=\frac{\left|X_{corrette}\right|}{\left|X\right|}\]
Ovviamente, con l'addestramento si vuole minimizzare questo rapporto. Quello descritto è un problema di ottimizzazione.\\

Tuttavia non basta che il modello si comporti bene su una collezione di dati di cui fondamentalmente si conosceva già l'output (abilità di per sé abbastanza inutile), ma si vuole, come già spiegato, che esso lavori bene anche su un \textbf{test set} $T$ di esempi mai forniti in input per l'addestramento del modello, e pertanto non presenti nel training set. Si può definire similmente al \textit{training error} un parametro detto \textbf{\textit{generalization error}} (errore di generalizzazione), o \textbf{\textit{test error}}.
\[\text{Test Error}=\frac{|T_{corrette}|}{|T|}\]
Si vuole ovviamente che anche il test error, come il training error, sia piccolo. Più precisamente, si vuole che la differenza tra il training error e il test error sia quanto più piccola possibile.\footnote{Il problema di ottimizzazione da risolvere è quello della minimizzazione del training error. Il test error non può essere minimizzato, in quanto esso viene valutato quando l'addestramento della rete è finito; anche in fase di training, comunque, il test set non è disponibile per l'addestramento (non può essere trattato come un'estensione del training set, pena la violazione della definizione stessa di "test set").}\\

I fattori che determinano quanto bene un algoritmo di machine learning performerà sono in definitiva le sue capacità di
\begin{enumerate}
\item rendere piccolo il training error
\item rendere piccola la differenza tra training error e test error
\end{enumerate}

Quando un modello non riesce ad ottenere un errore sufficientemente basso sul training set si dice che il modello soffre di \textbf{\textit{underfitting}} (sottoadattamento). Quando il modello non riesce a rendere piccola a sufficienza la differenza tra training error e test error si dice che il modello soffre di \textbf{\textit{overfitting}} (sovradattamento).

Possiamo controllare la tendenza di un modello all'underfitting o all'overfitting  regolando la sua \textbf{capacità}. Informalmente, la capacità di un modello è la sua abilità ad adattarsi ad un insieme ampio di funzioni. Modelli con capacità bassa potrebbero avere difficoltà nell'adattarsi al training set (underfitting). Al contrario, modelli con capacità alta hanno una maggiore probabilità di adattarsi troppo bene al training set (overfitting), memorizzando molte caratteristiche e proprietà degli esempi del training set che non sempre aiutano nella generalizzazione ai nuovi casi, ad esempio quelli del test set.

In una rete neurale (par. \ref{ANN}) la capacità del modello può essere definita, ad esempio, come il numero di parametri addestrabili (pesi e bias) che la caratterizzano.

Gli algoritmi di machine learning performeranno generalmente bene quando la loro capacità è appropriata rispetto alla reale complessità del task che devono svolgere e alla quantità di esempi $|X^{train}|$ forniti per l'addestramento. Modelli con una capacità insufficiente non riescono a risolvere task complessi. Modelli con alta capacità possono risolvere task complessi, ma se la loro capacità è eccessivamente alta rispetto alla complessità del task in esame potrebbero soffrire di overfitting. La fig. \ref{fig:capacity} presenta bene la situazione.

\begin{figure}[h]
\centering
\includegraphics[width=0.8\textwidth]{capacity}
\caption{Training error e test error (asse y) al variare della capacità del modello (asse x)}
\label{fig:capacity}
\end{figure}

Negli esperimenti condotti una delle reti neurali utilizzate (ResNet-50) ha sofferto di overfitting a causa della sua enorme capacità di rappresentazione e del training set relativamente ristretto usato per il suo addestramento (par. \ref{addestramento}).

\section{Classificatore lineare}
\label{classificatoreLineare}
Il classificatore lineare è uno tra i più semplici modelli di classificazione.
Ipotizziamo di avere un insieme di $N$ immagini $\mathbf{x}^{(i)}$ (\textit{data points}), ciascuna con risoluzione fissa $w\times h$ e in formato RGB ($c=3$), e un insieme di $K$ distinte categorie di oggetti  (\textit{labels}). Un \textbf{classificatore lineare} è definito dalla funzione
\begin{equation} \label{eq_class_lin}
f(\mathbf{x}^{(i)};\mathbf{W},\mathbf{b})=\mathbf{W}\mathbf{x}^{(i)}+\mathbf{b}
\end{equation}
In questa espressione stiamo assumendo che $\mathbf{x}^{(i)}$ sia un vettore colonna di dimensione $D=hwc$ ottenuto incolonnando una ad una le righe dell'$i$-esima immagine di tutti e tre i canali di colore, $\mathbf{W}$ una matrice detta \textbf{matrice dei pesi} (\textit{weights matrix}) di dimensione $K\times D$ e $\mathbf{b}$ un vettore colonna detto \textbf{vettore dei bias} (\textit{bias vector}) di dimensione $K$. I pesi e i bias sono parametri della funzione $f$.

Ogni riga $j$-esima di $W$ e il relativo $j$-esimo valore di $\mathbf{b}$ serve a calcolare la combinazione (lineare a meno del bias) $\mathbf{w}_j\cdot \mathbf{x}^{(i)}+b_j$. Ognuna delle $K$ combinazioni calcolate è un numero reale che si può interpretare come un "punteggio" registrato dall'$i$-esima immagine in ogni classe di oggetti in $Y$ (\textit{class score}): l'$i$-esima immagine è classificata con l'etichetta $\mathbf{y}_j\in Y$ se l'elemento $j$-esimo del vettore output $f(\mathbf{x}^{(i)};\mathbf{W},\mathbf{b})$ è il massimo del medesimo vettore.\\

L'esempio in figura \ref{class_lin} mostra la classificazione di un'immagine di un gatto con $\abs{Y}=3$ classi (\textit{gatto}, \textit{cane}, \textit{barca}). Per semplicità, l'immagine input è ipotizzata $2\times 2$ e composta da un unico canale di colore ($c=1$) (quindi $\mathbf{x}$, scritta come vettore colonna, è $4\times 1$).

\begin{figure}[h!]
\centering
\includegraphics[width=\textwidth ,keepaspectratio]{classificatore_lineare}
\caption{Mappatura di un'immagine ai punteggi di ogni classe mediante un classificatore lineare. Si noti che i pesi di $\mathbf{W}$ non costituiscono un buon set di parametri: il punteggio assegnato alla classe "cane" (sbagliata) è alto e quello totalizzato dalla classe "gatto" (corretta) è basso. Il classificatore "è convinto" di aver classificato l'immagine di un cane.}
\label{class_lin}
\end{figure}

Come si vedrà nel par. \ref{neuroni}, nelle reti neurali il classificatore lineare sarà utilizzato come un singolo "blocco da costruzione" per costruire una rete più grande.

\subsection*{Interpretare un classificatore lineare}

Poiché le immagini possono essere memorizzate come vettori colonna $hwc$-dimensionali, si possono immaginare le immagini di un dataset come dei punti nello spazio $\R^{hwc}$. Di conseguenza, il dataset può essere pensato come una collezione di punti multidimensionali. Ovviamente non possiamo visualizzare spazi con più dimensioni di $\R^{3}$, ma se immaginiamo di "comprimere" tutte le $hwc$ dimensioni in sole due dimensioni otteniamo una visualizzazione del tipo in figura \ref{visual_class_lin}.

\begin{figure}[h]
\centering
\includegraphics[width=\textwidth ,keepaspectratio]{visualizz_class_lin}
\caption{Visualizzazione di tre righe di un classificatore lineare, una per ciascuna delle classi "aereo", "auto", "cervo".}
\label{visual_class_lin}
\end{figure}

Le rette in figura devono in realtà essere pensate come degli iperpiani $(hwc-1)$-dimensionali, associati a ciascuna classe di $Y$ (cioè a ciascuna riga di $\mathbf{W}$ e $\mathbf{b}$), e il piano come lo spazio $\R^{hwc}$. Sussistono le seguenti interpretazioni geometriche:
\begin{itemize}
\item Le immagini sono dei punti nel piano. Ogni retta è il luogo dei punti che totalizzano un punteggio nullo per la classe associata a quella retta (la classe è scritta in figura accanto ad ogni retta). La freccia nella figura indica la direzione seguendo la quale i punti del piano aumentano (linearmente) il punteggio realizzato per quella classe.
\item Modificare i pesi di $\mathbf{W}$ significa regolare l'inclinazione delle rette (cioè ruotarle rispetto al punto di intercetta).
\item Modificare i bias di $\mathbf{b}$ significa regolare l'intercetta delle rette (cioè traslarle verticalmente).
\end{itemize}

Un altro modo di interpretare i pesi $\mathbf{W}$ può essere quello di far corrispondere ogni riga di $\mathbf{W}$ a un \textbf{prototipo} (in inglese \textbf{template}) per una delle classi. In questa interpretazione, il punteggio realizzato per ogni classe da un'immagine è ottenuto attraverso l'operazione di prodotto matriciale tra il prototipo della classe $j$ ($\mathbf{w}_j$) e l'immagine da classificare ($\mathbf{x}^{(i)}$).
Usando la terminologia introdotta, possiamo affermare che ciò che sta facendo il classificatore lineare è un'operazione di \textit{template matching}, dove i \textit{templates} sono oggetto di apprendimento da parte del classificatore\footnote{Si introdurranno gli algoritmi di apprendimento (supervisionato) nel paragrafo \ref{supervisedLearning}.}.

Ad esempio, analizziamo il dataset \textit{CIFAR-10} \cite{cifar10}. Esso contiene immagini $32\times 32$ ciascuna appartenente ad una di 10 classi. Visualizzando\footnote{Per i dettagli su come "visualizzare" i pesi si veda \url{https://it.mathworks.com/help/deeplearning/examples/visualize-activations-of-a-convolutional-neural-network.html}.} i pesi (e quindi i 10 templates) di un classificatore lineare addestrato su CIFAR-10 si ottengono i risultati in figura seguente:

\begin{figure}[h]
\centering
\includegraphics[width=\textwidth ,keepaspectratio]{templates}
\caption{Visualizzazione dei templates di un classificatore addestrato sul dataset CIFAR-10}
\label{templates}
\end{figure}

Si possono fare alcune interessanti osservazioni.

Ad esempio, il prototipo della classe "barca" è composto da molti pixel blu disposti perlopiù lungo i margini, come ci si potrebbe aspettare dal momento che molte immagini di barche in CIFAR-10 raffigurano queste in mare aperto. Questo template allora assegnerà un punteggio alto quando l'immagine che si vuole classificare (cioè \textit{raffrontare al template}) è una barca in mare aperto. In altre parole, un'immagine realizzerà un punteggio tanto più alto in una certa classe quanto più essa è \textit{simile} al template che il classificatore lineare \textit{ha imparato} per quella classe.

Il prototipo per la classe "cavallo" sembra essere l'immagine di un cavallo a due teste; similmente, quello per la classe "auto" sembra una miscela di rappresentazioni di un'auto vista da più direzioni diverse. Ciò è coerente col fatto che il classificatore lineare è stato addestrato su immagini di cavalli visti rispetto a entrambi i profili e su immagini
di auto raffigurate in tante direzioni diverse. Inoltre, il template per l'auto sembra rappresentare un'auto di colore rosso: evidentemente in CIFAR-10 la maggior parte delle automobili rappresentate sono di quel colore.\\

Come si vedrà nel seguito, questa operazione di \textit{template matching} presenta una forte analogia con il funzionamento di un \textit{Fully Connected Layer} di una rete neurale convoluzionale.

\subsection*{Bias trick}
Concludiamo questo capitolo menzionando un modo molto utilizzato per rappresentare $\mathbf{W}$ e $\mathbf{b}$ come un'unica matrice, semplificando la notazione \ref{eq_class_lin}.
Possiamo aggiungere il vettore dei bias in coda alla matrice dei pesi e aggiungere un "1" in coda al vettore che rappresenta l'immagine. In questo modo, il classificatore lineare è rappresentato dalla funzione di associazione
\begin{equation} \label{eq_bias_trick}
f(\mathbf{x}^{(i)};\mathbf{W})=\mathbf{W}\mathbf{x}^{(i)}
\end{equation}

In questa maniera, $f$ calcola solo combinazioni lineari (un singolo prodotto matriciale), poiché il vettore dei bias è stato eliminato.
Tale utile passaggio, noto come \textit{bias trick}, è visualizzato nella seguente figura

\begin{figure}[h]
\centering
\includegraphics[width=\textwidth ,keepaspectratio]{bias_trick}
\caption{Bias trick}
\label{bias_trick}
\end{figure}



\section{Reti Neurali}
\label{ANN}

Le reti neurali artificiali (ANN, \textit{artificial neural networks}), costituiscono i modelli di deep learning per eccellenza.

Una rete neurale può essere interpretata come un insieme di strati (\textit{layer}) composti
ciascuno da un certo numero di unità computazionali, dette neuroni, in grado di fornire una nuova rappresentazione dell'input, secondo il paradigma del representation learning. Le funzioni sono composte a formare una catena di rappresentazioni sconosciute (per questo dette \textit{hidden layers}), nel senso che ciascun layer calcola una funzione dell'output del layer precedente: a partire da rappresentazioni più semplici, esse vengono raggruppate fino ad un livello di "arbitraria" complessità
\begin{equation*}
\mathcal{F}(\mathbf{x})=f^{(d)}\underbrace{(\dots(\mathbf{h}^{(3)}(\mathbf{h}^{(2)}(\mathbf{h}^{(1)}(\mathbf{x})))))}
\end{equation*}
in cui
\begin{itemize}
\item $d$ è la profondità della rete, cioè il numero di layers che la compongono;
\item $\mathbf{h}^{(1)}$ è il primo (hidden) layer, $\mathbf{h}^{(2)}$ il secondo, e così via. Ciascuno di essi rappresenta una trasformazione parametrica in generale non lineare delle feature relative ad un input $\mathbf{x}$: ogni hidden layer $\mathbf{h}$ accetta un vettore in input $\mathbf{x}$, calcola una
trasformazione affine $\mathbf{z}=\mathbf{W}\mathbf{x}+\mathbf{b}$, quindi applica una funzione non lineare $g(\mathbf{z})$
elemento per elemento, detta \textbf{funzione di attivazione}. Si ottiene così:
\begin{equation*}
\text{Layer 1:   } \mathbf{h}^{(1)}=g^{(1)}\left(\mathbf{W}^{(1)}\mathbf{x}+\mathbf{b}^{(1)}\right)

\text{Layer 2:   } \mathbf{h}^{(2)}=g^{(2)}\left(\mathbf{W}^{(2)}\mathbf{h}^{(1)}+\mathbf{b}^{(2)}\right)

\text{Layer 3:   } \mathbf{h}^{(3)}=g^{(3)}\left(\mathbf{W}^{(3)}\mathbf{h}^{(2)}+\mathbf{b}^{(3)}\right)
\end{equation*}
fino a giungere all'output layer:
\begin{equation*}
\text{Layer d:   } \mathbf{f}^{(d)}=g^{(d)}\left(\mathbf{W}^{(d)}\mathbf{h}^{(d-1)}+\mathbf{b}^{(d)}\right)
\end{equation*}

\item $f^{(d)}$ è l'output layer, che ha il ruolo di fornire un'ultima trasformazione al fine di completare il task che la rete deve eseguire. Le scelte solitamente sono:
\begin{itemize}
\item Layer di output lineare: viene calcolata una ulteriore trasformazione affine, del tipo
\begin{equation*}
\mathbf{\widehat{y}}=\mathbf{W}\mathbf{h}^{(d-1)+\mathbf{b}
\end{equation*}
\item Layer di output sigmoide: viene applicata una trasformazione affine, quindi usata la funzione sigmoide $\sigma$ per convertire il risultato in una probabilità:
\begin{equation*}
\widehat{y}=\sigma\left(\mathbf{W}\mathbf{h}^{(d-1)+\mathbf{b}\right)
\end{equation*}
Questo approccio è usato nel caso di classificazione binaria, come descritto al par. \ref{classificazione}.
\item Layer di output softmax: viene calcolata una trasformazione affine
\begin{equation*}
\mathbf{z}=\mathbf{W}\mathbf{h}^{(d-1)+\mathbf{b}
\end{equation*}
e viene quindi applicata la funzione softmax (par. \ref{classificazione})
\begin{equation*}
\mathbf{\widehat{y}}=\text{softmax}(\mathbf{z})_i=frac{e^{z_i}}{\sum\nolimits_{j}e^{z_j}}
\end{equation*}
Grazie alla funzione softmax, l'output $\mathbf{\widehat{y}}$ è la distribuzione di probabilità che $\mathbf{x}$ appartenga alla classe $i$:
\begin{equation*}
\widehat{y_i}=p(y=i|\mathbf{x})
\end{equation*}
\end{itemize}
\end{itemize}

La funzione di attivazione più utilizzata nell'ambito delle ANN è la seguente \textbf{ReLU} (Rectified Linear Unit), anche detta \textbf{rettificatore}:
\[\text{ReLU}(x)=x^{+}=\max(0,x)\]
Il rettificatore è la funzione di attivazione usata in tutte le reti neurali presentate in questo lavoro di tesi (par. \ref{alexnet},\ref{googlenet},\ref{resnet}).
Altre possibilità sono la sigmoide $g(x)=\sigma(x)$ oppure la tangente iperbolica $g(x)=\tanh(x)$.\\

Anziché pensare ad un layer come ad una singola funzione vettoriale, possiamo pensare il layer costituito da tante unità computazionali che agiscono in parallelo,
ognuna delle quali calcola la propria funzione scalare: esegue una somma pesata
degli input con i pesi ed applica un bias (fig. \ref{fig:modelloHebb}) nella misura in cui riceve input da molte altre unità (dendriti) e calcola una
funzione di attivazione trasmessa come output ad altre unità del layer successivo (assone).
La scelta delle funzioni f(i)(x) è generalmente legata ad una rappresentazione
molto semplificata della funzione che il neurone biologico calcola.


\section{AlexNet}\label{alexnet}
AlexNet è una CNN creata tra il 2011 e il 2012 da Alex Krizhevsky, in collaborazione con Ilya Sutskever e Geoffrey Hinton \cite{alexnet}. La vittoria di AlexNet nella ImageNet Large Scale Visual Recognition Challenge (ILSVRC) \cite{imagenet2012}, ottenuta con un netto distacco nei confronti degli altri concorrenti, ha segnato l'inizio dell'enorme successo ottenuto dalle reti neurali profonde in svariati domini di applicazione \cite{historydl}

Il risultato principale di AlexNet, così come dichiarato dai suoi creatori nell'articolo originale, è il fatto che la profondità del modello è stato essenziale per conferirgli prestazioni così alte. L'alto costo computazionale dell'addestramento di AlexNet, reso oneroso appunto dalla profondità del modello (e quindi dal grande numero di parametri - circa 62.3 milioni) è stato affrontato con l'impiego di schede grafiche (GPU), che cominciavano in quegli anni a raggiungere notevoli potenze di calcolo.

\subsection{Architettura di AlexNet}
L'architettura di AlexNet è riportata schematicamente nella figura \ref{arc_alexnet} e con maggiore dettaglio in tabella \ref{tab_arc_alexnet}.
La rete accetta in input immagini $227\times 227$. Essa si compone di otto layer con parametri - cinque convoluzionali e tre completamente connessi. L'output dell'ultimo layer completamente connesso passa per un softmax layer a 1000 vie, il quale fornisce la distribuzione di probabilità per le 1000 classi del dataset ImageNet.

Tra ognuno degli otto strati parametrizzati sono interposti alcuni strati intermedi: ReLU layer, Local Response Normalization layer, Max Pooling layer, Dropout layer. Ognuno di questi sarà analizzato in maggiore dettaglio nei paragrafi successivi.

\begin{figure}[h]
\centering
\includegraphics[width=\textwidth ,keepaspectratio]{arc_alexnet}
\caption{Architettura originale di AlexNet \cite{alexnet}}
\label{arc_alexnet}
\end{figure}

Come si evince dalla figura \ref{arc_alexnet}, la rete è composta da due "\textit{pipeline}" parallele. Si scelse infatti di "estendere" la rete su due GPU NVIDIA\textsuperscript{\textregistered} GeForce\textsuperscript{\textregistered} GTX 580 3GB in fase di training, per raddoppiare la memoria massima disponibile (6GB in totale) per conservare la rete e i suoi parametri.

Queste GPU si prestano bene a lavorare in parallelo, poiché possono leggere e scrivere l'una sull'altra direttamente, senza passare dalla memoria della macchina host. Lo schema di parallelizzazione a due vie prevede che su ogni GPU risieda la metà dei kernel (o dei neuroni) di ciascuno strato parametrizzato. Le GPU possono comunicare tra loro solo in certi strati. In particolare, i kernel del layer convoluzionale 1 e 3 hanno in input l'intero output volume rispettivamente del layer di input e del layer convoluzionale 2, mentre i kernel dei rimanenti strati convoluzionali hanno in input la sola metà dell'output volume presente nella stessa GPU (\textit{grouped convolution}\footnote{La scelta di questo pattern di connettività fra le due GPU parallele è il risultato di un problema di cross-validation.}.\\

Sono di seguito passate in rassegna le principali scelte architetturali introdotte in AlexNet, ed alcuni dettagli relativi al suo addestramento.

\subsection*{Funzione di attivazione ReLU}
Dopo ogni strato parametrizzato, i valori delle attivazioni sono passati alla funzione attivatrice "rettificatore": $f(x)=x^{+}=\max(0,x)$ \cite{nairhinton}. Questa funzione attivatrice non-lineare e non soggetta a saturazione permette un addestramento molto più veloce delle reti convoluzionali profonde, in confronto a funzioni attivatrici fino ad allora più utilizzate come la funzione sigmoidea $f(x)=(1+\exp^{-x})^{-1}$ e la funzione tangente iperbolica $f(x)=\tanh(x)$.

\subsection*{Local Response Normalization}
È stato verificato che la seguente normalizzazione delle attivazioni, \textit{Local Response Normalization}, aumenta lievemente la capacità di generalizzazione del modello:

\[b_{x,y}^{i}=a_{x,y}^{i}/\left(k+\alpha \sum_{j=\max(0,i-n/2)}^{\min(N-1,i+n/2)}(a_{x,y}^{j})^{2}\right)^{\beta}\]

dove $a_{x,y}^{i}$ è l'attivazione del neurone ottenuto applicando il kernel $i$-esimo alla posizione $(x,y)$ e applicando in seguito la funzione ReLU, $b_{x,y}^{i}$ l'attivazione normalizzata, $N$ il numero totale di kernel del layer corrente, $k, n, \alpha, \beta$ sono iperparametri; sono stati usati i valori $k=2, n=5, \alpha=10^{-4}, \beta=0.75$.

Questa normalizzazione è adoperata solamente nel primo e nel secondo layer convoluzionale.

\subsection*{Overlapping Max Pooling}
La funzione di max pooling in AlexNet è stata caratterizzata dalla scelta di una dimensione del filtro di pooling $3\times 3$ e uno stride di $2$. È stato osservato durante la fase di training che questa funzione di \textit{max pooling con sovrapposizione} ha attenuato lievemente l'\textit{overfitting} della rete.

\subsection*{Data Augmentation}
Una delle difficoltà che si incontrano spesso quando si vuole addestrare una rete neurale con moltissimi parametri avendo a disposizione un dataset relativamente piccolo è il rischio del sovradattamento (\textit{overfitting}) della rete al training set, che compromette anche seriamente le prestazioni della rete quando le vengono presentati nuovi dati.
In AlexNet l'overfitting è stato ridotto grazie a tecniche di \textit{data augmentation}. In particolare, dopo aver ridimensionato a $256\times 256$ tutte le immagini del training set, quest'ultimo è stato "arricchito" con le seguenti immagini:
\begin{itemize}
\item Estrazione casuale di ritagli $224\times 224$ dalle immagini
\item Riflessione orizzontale ("a specchio") delle immagini
\item Somma di un'immagine e le sue componenti principali (PCA)\footnote{L'\textit{analisi delle componenti principali} (PCA, principal component analysis) è una tecnica per la semplificazione dei dati utilizzata nell'ambito della statistica multivariata. In questa sede ci limitiamo a specificare che il suo utilizzo nell'ambito della data augmentation è di evidenziare una importante proprietà delle immagini naturali, e cioè che l'identità di un oggetto è invariante rispetto ai cambi d'intensità e di colori nella sua illuminazione. Si rimanda ad esempio a \cite{PCA} per approfondimenti sulla PCA.}
\end{itemize}

\subsection*{Dropout}
Un altro modo per ridurre il problema del sovradattamento è l'impiego di tecniche di regolarizzazione dei parametri. AlexNet utilizza la tecnica del \textit{dropout} \cite{dropout}. Questa tecnica consiste nel settare a zero l'attivazione di ciascun neurone di un layer intermedio con probabilità $p$ (AlexNet impiega un dropout con $p=0.5$. I neuroni "azzerati" sono essenzialmente eliminati dalla rete e non contribuiscono né alla propagazione all'indietro del gradiente né al calcolo delle attivazioni nello strato finale (in fase di addestramento). Questa tecnica riduce il \textit{co-adattamento} tra neuroni: ogni neurone non può fare affidamento sulla presenza di altri neuroni, ed è costretto ad apprendere feature utili in congiunzione con diversi sottoinsiemi casuali degli altri neuroni, e non con un solo particolare sottoinsieme, migliorando la generalizzazione su nuovi dati.

In AlexNet, il dropout dei neuroni è utilizzato nei primi due layer completamente connessi. In fase di test, i neuroni di questi due strati sono moltiplicati per 0.5 per tenere conto dell'impiego del dropout in addestramento.

\subsection*{Addestramento di AlexNet}
Nella sua forma originale, AlexNet fu addestrato usando la discesa stocastica del gradiente con momento$=0.9$, mini-batch$=128$ e decadimento dei pesi (weight decay) $=0.0005$. Dettagli più specifici sulla fase di addestramento di AlexNet possono essere trovati nel paper originale \cite{alexnet}.

\begin{table}[h]
\caption{Architettura originale di AlexNet}
\label{tab_arc_alexnet}
\begin{tabularx}{\textwidth}{@{}llll@{}}
\toprule
N & Layer           & Attivazioni & Parametri \\ \midrule
1  &
INPUT &
$(227\times 227\times 3)$ &
\\ \midrule
2  & CONVOLUTION     & $(55\times 55\times 96)$ &\acapo{Pesi: $(11\times 11\times 3)\times 96$\\Bias: $(96)$} \\
3  & RELU            & --            & --\\
4  & NORMALIZATION   & --            & --\\
5  & MAX POOLING     & $(27\times 27\times 96)$ & -- \\ \midrule
6  & GROUPED CONVOLUTION     & $(27\times 27\times 256)$ & \acapo{Pesi: $(5\times 5\times 48)\times 128\times 2$\\Bias: $(128)\times 2$} \\
7  & RELU            & --            & --          \\
8  & NORMALIZATION   & --            & --          \\
9  & MAX POOLING     & $(13\times 13\times 256)$            &--           \\ \midrule
10 & CONVOLUTION     & $(13\times 13\times 384)$            & \acapo{Pesi: $(3\times 3\times 256)\times 384$\\Bias: $(384)$}          \\
11 & RELU            & --            &   --        \\ \midrule
12 & GROUPED CONVOLUTION     & $(13\times 13\times 384)$ & \acapo{Pesi: $(3\times 3\times 192)\times 192\times 2$\\Bias: $(192)\times 2$}  \\
13 & RELU            & --            &       --    \\ \midrule
14 & GROUPED CONVOLUTION     & $(13\times 13\times 256)$            & \acapo{Pesi: $(3\times 3\times 192)\times 128\times 2$\\Bias: $(128)\times 2$}\\
15 & RELU            & --            &   --        \\
16 & MAX POOLING     &$(6\times 6\times 256)$            &     --      \\ \midrule
17 & FULLY CONNECTED &$4096$& \acapo{Pesi: $4096\times 9216$\\Bias: $4096$} \\ 
18 & RELU            & --            &     --      \\
19 & DROPOUT         & --            &    --       \\ \midrule
20 & FULLY CONNECTED &$4096$&\acapo{Pesi: $4096\times 4096$\\Bias: $4096$}\\
21 & RELU            & --            &   --        \\
22 & DROPOUT         & --            &   --        \\ \midrule
23 & FULLY CONNECTED &$1000$&\acapo{Pesi: $2\times 4096$\\Bias: $2$}\\
24 & SOFTMAX         & --            &    --       \\
25 & CROSS-ENTROPY LOSS  & --            &   --     \\ \bottomrule                 
\end{tabularx}
\end{table}


\section{GoogLeNet}
\label{googlenet}
GoogLeNet (in origine \textit{Inception v1}) è una rete neurale convoluzionale profonda presentata nel 2014 da Christian Szegedy, ricercatore presso Google, ed il suo team di ricerca. La pubblicazione del paper originale \cite{googlenet} è avvenuta poco dopo la schiacciante vittoria riportata da GoogLeNet nella \textit{ILSVRC 2014}\footnote{\url{http://image-net.org/challenges/LSVRC/2014/}}, sia nel problema di \textit{object classification} che in quello di \textit{object detection}, introdotto nell'anno precedente.

Il grande contributo di GoogLeNet nell'ambito  del deep learning è il miglioramento nell'utilizzo delle risorse computazionali da parte di una rete profonda, grazie all'introduzione del modulo \textit{Inception} (par. \ref{inception}. In particolare, se confrontata ad AlexNet (par. \ref{alexnet}), GoogLeNet utilizza circa 10 volte meno parametri (circa 6,8 milioni) essendo però significativamente più profonda (22 layer con parametri) e performante (errore \textit{top-5} su dataset ImageNet 6,7\%).\\

GoogLeNet è stata progettata per risolvere alcuni problemi tipici delle reti convoluzionali particolarmente profonde:

\begin{itemize}

\item L'operazione di convoluzione su volumi molto profondi è computazionalmente onerosa. Per risolvere questo problema, GoogLeNet introduce l'operazione di convoluzione $1\times 1$ (par. \ref{1x1conv}).

\item Le reti neurali molto profonde sono molto sensibili al rischio di \textit{overfitting}, a causa del loro alto grado di astrazione e rappresentazione dei concetti (dovuto al grandissimo numero di parametri). Questo problema è attenuato grazie all'introduzione della già citata operazione di convoluzione $1\times 1$, all'operazione di \textit{global average pooling} (par. \ref{globalAveragePooling}) e all'usuale pratica della \textit{image augmentation} (par. \ref{googlenetAugmentation}).

\item Le parti salienti di un'immagine possono avere delle dimensioni estremamente variabili all'interno di essa. Si guardi ad esempio la figura seguente: 

\begin{figure}[h!] 
\centering
\includegraphics[width=0.9\textwidth]{cani.png}
\caption{Tre immagini di cani. L'area occupata da ciascun cane è differente e via via più piccola in ogni immagine}
\label{fig:cani}
\end{figure}

A causa di questa grande variabilità nella localizzazione dell'informazione la scelta delle dimensioni dei filtri convoluzionali diventa complessa. Un filtro più ampio è preferito quando l'informazione è distribuita su un'area vasta dell'immagine; un filtro più piccolo è adeguato in quei casi in cui l'informazione è localizzata in un'area più ristretta.\\
Questo problema è risolto con l'introduzione dei moduli \textit{inception} (par. \ref{inception}).

\item A causa della profondità di una rete, un altro tipico problema che si presenta durante l'addestramento è la scomparsa del gradiente (\textit{vanishing gradient problem}, par. \ref{vanishingGradient}), nella fase di \textit{backpropagation}. Questo problema è risolto prevedendo una particolare architettura della rete, che in fase di addestramento risulta in realtà composta da tre sottoreti diverse (par. \ref{classificatoriAusiliari}).

\end{itemize}


\subsection{Convoluzione $1\times 1$}
\label{1x1conv}
Mutuando un'idea precedentemente esposta in \cite{NiN}, GoogLeNet introduce l'operazione di convoluzione $1\times 1$ (seguita dalla funzione di attivazione ReLU). Lo scopo di questa operazione è quello di ridurre le dimensioni (in particolare, la profondità) di un volume di attivazioni per ridurre il costo computazionale delle operazioni di convoluzione successive. Questo permette di ottenere reti più profonde ma anche - con il modulo \textit{inception} (par. \ref{inception}) - più "larghe".
La fig. \ref{fig:1x1conv} mostra schematicamente l'applicazione di un'operazione di convoluzione $1\times 1$.

\begin{figure}[h] 
\centering
\includegraphics[width=0.85\textwidth]{1x1conv.png}
\caption{Convoluzione $1\times 1$ applicata ad un volume di attivazioni $6\times 6\times 32$. La profondità del volume di output dipende dal n. di filtri $1\times 1$ adoperati. In questo caso si ottiene una riduzione della profondità per $n<32$.}
\label{fig:1x1conv}
\end{figure}

Grazie all'introduzione della convoluzione $1\times 1$, inoltre, il numero di pesi nei kernel delle operazioni di convoluzione successive diminuiscono, riducendo così il rischio di overfitting.\\

A conclusione del paragrafo viene riportata un'osservazione interessante. Le reti neurali artificiali "standard" (contenenti cioè solo strati completamente connessi) possono essere convertite in reti convoluzionali mediante sole convoluzioni $1\times 1$. Infatti nelle ANN con soli strati completamente connessi tutti i volumi su cui si opera hanno dimensione $1\times 1\times k$, dove $k$ è variabile ed è il numero $k$ di neuroni dello strato che ha generato tale volume; si può allora pensare a ciascuno di questi strati completamente connessi con $k$ neuroni (biunivocamente associati al volume che generano in output) come ad uno strato di convoluzione $1\times 1$ che applica $k$ filtri al volume in input. Ogni neurone del volume di output $1\times 1\times n$ ha così una completa connettività con tutti i neuroni del volume (e quindi dello strato completamente connesso) precedente.

\subsection{Modulo \textit{Inception}}
\label{inception}
Per ovviare al problema della variabilità dell'area occupata da un oggetto da classificare in un'immagine, l'idea chiave è stata quella di operare molteplici convoluzioni, con diverse dimensioni dei kernel, in parallelo.
Così facendo la rete tende a diventare più "larga" anziché più "profonda".
Il modulo \textit{inception} è stato sviluppato proprio sulla base di questa idea.
In fig. \ref{fig:inception} è mostrato il modulo \textit{inception}, innestato più volte nell'architettura di GoogLeNet (par. \ref{architetturaGooglenet})

\begin{figure}[h] 
\centering
\includegraphics[width=\textwidth]{inception.jpg}
\caption{Modulo \textit{inception}}
\label{fig:inception}
\end{figure}

Tale modulo riceve un volume di input da uno strato precedente e opera in parallelo su di esso tre convoluzioni (con filtri $1\times 1$, $3\times 3$, $5\times 5$ rispettivamente) e un \textit{max pooling} con finestra $3\times 3$. Per diminuire il costo computazionale delle operazioni di convoluzione, si è deciso di implementare una convoluzione $1\times 1$ prima delle convoluzioni $3\times 3$ e $5\times 5$ e dopo il \textit{max pooling} $3\times 3$.
Gli output sono infine "impilati" tra loro lungo la dimensione della profondità, e il volume risultante viene inviato al successivo layer(o modulo \textit{inception}).

\subsection{Classificatori ausiliari}
\label{classificatoriAusiliari}
L'architettura di GoogLeNet prevede, per la sola fase di addestramento, delle "diramazioni", posizionate circa a metà della rete, evidenti in fig. \ref{architetturaGooglenet}.
Questi rami della rete ospitano dei classificatori ausiliari che consistono di
\begin{itemize}
\item Average Pooling $5\times 5$ (Stride 3)
\item Convoluzione $1\times 1$ (128 filtri)
\item ReLU layer
\item 1024 Fully Connected layer
\item 1000 Fully Connected layer
\item Dropout layer (70\% di probabilità di dropout di ciascun output; vd. par. \ref{dropout})
\item Softmax layer
\end{itemize}
Questa particolare architettura è stata ideata per attenuare il problema della scomparsa del gradiente (\textit{vanishing gradient problem}, par. \ref{vanishingGradient}), tipico delle reti molto profonde. L'idea si basa su un'osservazione: le promettenti performance di reti meno profonde nei task di \textit{image classification} suggeriscono che le \textit{feature} estratte dagli strati intermedi della rete hanno un grande impatto sulla predizione finale. Pertanto, nel tentativo di propagare meglio il segnale gradiente all'indietro (cioè per amplificarlo), la funzione costo calcolata è amplificata sulla base delle predizioni dei classificatori intermedi.
In fase di addestramento, dai valori di output dello strato softmax dei tre classificatori (i due intermedi e quello finale) viene calcolata la funzione costo (\textit{cross-entropy loss}). 
Il costo totale sarà costituito dalla somma pesata del costo registrato dal classificatore finale (con peso 1) e quello registrato dai due classificatori ausiliari (con peso $0.3$).

\subsection{Global Average Pooling}
\label{globalAveragePooling}
In precedenza, nella parte conclusiva delle architetture delle reti convoluzionali venivano posti due o più layer completamente connessi (ad esempio in AlexNet, par. \ref{alexnet}, che presenta tre fully connected layer finali).
Mutuando un'idea esposta in \cite{NiN}, in GoogLeNet si è previsto invece un singolo layer completamente connesso finale con 1000 neuroni (quello che fornisce valori alla funzione softmax) preceduto da uno strato di pooling denominato \textit{global average pooling}. Questo strato opera un pooling su una superficie $7\times 7$ (le dimensioni di base e altezza del volume di output precedente) e restituisce una superficie $1\times 1$, il cui valore è la media calcolata sui 49 valori dei neuroni della superficie. Gli autori hanno mostrato che passare da un fully connected layer a un average pooling layer ha contribuito a migliorare la \textit{top-1 accuracy} nella ILSVRC del 0.6\%, abbassando ulteriormente il numero dei pesi della rete (da circa 51.2 milioni se fosse stato utilizzato il FC layer agli 0 del global average pooling). In questo modo, GoogLeNet è stata resa anche leggermente più robusta all'\textit{overfitting}.
La fig. \ref{fig:FCvsGAP} mostra un confronto tra il funzionamento di un FC layer e un GAP (Global Average Pooling) layer.

\begin{figure}[h]
\centering
\includegraphics[width=0.85\textwidth]{FCvsGAP.png}
\caption{Confronto fra un FC layer e un GAP layer}
\label{fig:FCvsGAP}
\end{figure}

\subsection{Data Augmentation}
\label{googlenetAugmentation}
Per ridurre ulteriormente l'overfitting al training set di ImageNet, GoogLeNet fa uso di una particolare tecnica di \textit{data augmentation} (par. \ref{augmentation}). Al posto delle immagini originali, per l'addestramento sono stati utilizzati dei ritagli (uno per ciascuna immagine di un mini-batch) di dimensioni distribuite uniformemente tra l'8\% e il 100\% dell'intera immagine e con \textit{aspect ratio} (rapporto tra larghezza e altezza dell'immagine) distribuito uniformemente tra 3/4 e 4/3. In aggiunta, sono state operate alcune distorsioni fotometriche già adoperate in precedenza in \cite{photometric}.

\subsection{Architettura di GoogLeNet}
\label{architetturaGooglenet}
L'architettura di GoogLeNet è riportata in forma tabellare in tab. \ref{tab:tabellaGooglenet} e in forma grafica in fig. \ref{fig:googlenet}

\begin{figure}[h]
\centering
\includegraphics[width=0.9\textwidth]{tabellaGooglenet.png}
\caption{Architettura di GoogLeNet}
\label{tab:tabellaGooglenet}
\end{figure}

\begin{figure}[tb] 
\centering
\includegraphics[width=0.95\textwidth, height=0.99\textheight, keepaspectratio]{GoogLeNet.png}
\caption{Architettura di GoogLeNet}
\label{fig:googlenet}
\end{figure}

La rete accetta in input immagini $224\times 224$. I primi layer parametrizzati dall'inizio della rete sono 3 layer convoluzionali, a cui seguono 9 moduli \textit{inception} (ognuno con 6 layer convoluzionali) e infine un fully connected layer (finale).
Si evidenzia nuovamente, come già esposto nel par. \ref{classificatoriAusiliari}, che le ramificazioni che ospitano i due classificatori ausiliari esistono solo in fase di addestramento, e vengono eliminati dopo di esso.

\subsection{Addestramento di GoogLeNet}
GoogLeNet fu addestrato sul dataset di ImageNet (par. \ref{imagenet}) su una macchina che usava solamente le sue CPU per effettuare calcoli\footnote{Tuttavia gli autori affermano che se fosse stato usato un ridotto numero di GPU di fascia alta l'addestramento avrebbe potuto essere effettuato in meno di una settimana (avendo come unica limitazione la memoria delle GPU stesse).}, usando come algoritmo di ottimizzazione la discesa stocastica del gradiente con momento = 0.9, \textit{learning rate}\footnote{Per la ILSVRC 2014, il team ha in realtà utilizzato un ensemble di 7 reti GoogLeNet diverse, ognuna addestrata con \textit{learning rate} iniziale e \textit{mini-batch size} diversi} con \textit{annealing} del 4\% ogni 8 epoche. Dettagli più specifici sulla fase di addestramento di GoogLeNet possono essere trovati nel paper originale \cite{googlenet}.
\section{ResNet}
\label{resnet}
ResNet (abbreviazione di \textit{residual network}) è il nome di un'ampia categoria di reti neurali convoluzionali profonde, caratterizzate dall'utilizzo di particolari blocchi di layer detti \textit{residual blocks}, ideati per risolvere alcuni problemi delle reti molto profonde.
Le reti ResNet propriamente dette sono state introdotte da un gruppo di ricercatori Microsoft nel 2015 nel paper \cite{resnet}. Presentate nell'edizione del 2015 della \textit{ILSVRC}\footnote{\url{http://image-net.org/challenges/LSVRC/2015/results}} e risultate vincitrici con uno strabiliante errore \textit{top-5} su dataset ImageNet 3.57\% (più basso di quello umano, stimato a 5\%), queste reti sono considerate tutt'oggi lo stato dell'arte nell'ambito delle reti neurali convoluzionali profonde.\\

Nell'ambito dei problemi di visione artificiale proposti nella ILSVRC, in quegli anni cominciava ad essere evidente (\cite{googlenet},\cite{verydeep}) che la profondità delle reti neurali era di fondamentale importanza per il miglioramento delle prestazioni delle reti su dataset di grandi dimensioni, quali il database ImageNet (par. \ref{imagenet}).
L'ovvia conseguenza di questa osservazione è stato il tentativo di aumentare progressivamente il numero di layer delle reti neurali, rendendole sempre più profonde e complesse, con l'aggravarsi di ostacoli già noti (quali ad esempio il problema della scomparsa del gradiente, par. \ref{vanishingGradient}) e il presentarsi di nuovi (un problema di degradazione esposto per la prima volta in \cite{highway}, e descritto nel seguito):
ResNet nasce per dare soluzione a questi problemi:

\begin{itemize}

\item Il problema della scomparsa e dell'esplosione del gradiente è attenuato da un insieme di strategie già messe in campo in precedenza, su tutte la \textit{batch normalization} (par. \ref{batchNormalization}) introdotta dal team dei creatori di GoogLeNet \cite{batchNorm}.

\item Quando una rete molto profonda comincia a convergere verso un minimo della funzione costo (in fase di addestramento), si presenta un problema di degradazione: al crescere della profondità della rete la sua \textit{training accuracy} tende a saturarsi e in seguito prende a degradarsi rapidamente, come mostrato in fig. \ref{fig:degradation}.

\begin{figure}[h!]
\centering
\includegraphics[width=0.5\textwidth]{degradation.png}
\caption{Problema di degradazione della \textit{training accuracy} su CNN "standard" da 20 e 56 layer rispettivamente. La rete più profonda ha un \textit{training error} più alto. \cite{resnet} per più dettagli.}
\label{fig:degradation}
\end{figure}

È un problema inaspettato e diverso rispetto all'\textit{overfitting} (che interessa solamente il valore della \textit{validation accuracy}) e indica che la difficoltà nell'ottimizzare una rete convoluzionale "standard" cresce con la sua profondità. ResNet aggira questo problema con l'introduzione dei \textit{residual blocks} (par. \ref{residualBlock})

\end{itemize}

\subsection{\textit{Batch Normalization}}
\label{batchNormalization}
La \textit{Batch Normalization} è il nome di una tecnica introdotta dal team di GoogLeNet e ad oggi ampiamente utilizzata che permette un addestramento più veloce e stabile delle reti neurali profonde \cite{batchNorm}.

Essa consiste in una normalizzazione delle attivazioni di un certo layer, generalmente prima di passare gli stessi ad un'eventuale funzione di attivazione ed in seguito all'eventuale layer successivo.

L'algoritmo per l'applicazione della tecnica è riportato di seguito\\

\begin{adjustwidth}{3em}{0em}

\textbf{Input:} Attivazioni $\mathbf{x}=\{x_1,\dots,x_m\};$\\
\phantom{\textbf{Input:} }Parametri da imparare: $\gamma, \beta$\\
\noindent \textbf{Output:} Attivazioni normalizzate $y_i = BN_{\gamma,\beta}(x_i)\}$

\begin{align}
& \mu_{\mathbf{x}}\leftarrow\frac{1}{m}\sum_{i=1}^m x_i && \text{// media delle attivazioni} \notag\\
& \sigma^2_{\mathbf{x}}\leftarrow\frac{1}{m}\sum_{i=1}^m \left(x_i-\mu_{\mathbf{x}}\right)^2 && \text{// varianza delle attivazioni} \notag\\
& \widehat{x_i}\leftarrow\frac{x_i-\mu_{\mathbf{x}}}{\sqrt{\sigma^2_{\mathbf{x}}+\varepsilon}} && \text{// normalizzazione} \notag\\
& y_i\leftarrow\gamma\widehat{x_i}+\beta\equiv BN_{\gamma,\beta}(x_i)\} && \text{// scale e offset} \notag
\label{eq:batchNorm}
\end{align}

\end{adjustwidth}

I parametri $\gamma$ e $\beta$, chiamati rispettivamente \textit{scale} e \textit{offset}, sono imparati dalla rete in fase di addestramento. La loro funzione è far sì che le attivazioni normalizzate non siano necessariamente a media nulla e varianza unitaria (in modo da poter variare arbitrariamente l'intervallo utilizzato nel dominio della eventuale funzione di attivazione successiva alla normalizzazione).\\

Nonostante sia oggi ampiamente utilizzata, le ragioni dell'efficienza della \textit{batch normalization} sono ancora scarsamente comprese. Ricerche recenti \cite{realBatchNorm} hanno mostrato che ciò che questa tecnica produce non è una riduzione dell'\textit{internal covariate shift} (la variabilità statistica dei mini-batch usati, che ad ogni iterazione causa uno spostamento del punto di minimo della funzione costo), come affermato dagli autori del paper originale \cite{batchNorm}, ma è una "lisciatura" (\textit{smoothing}) della funzione costo, che induce un comportamento più stabile e predicibile dei gradienti, comportando un addestramento più veloce.

In ogni caso, la \textit{batch normalization} si è rivelata efficace per l'attenuazione del problema della scomparsa del gradiente o della sua esplosione; inoltre si è verificato che il suo utilizzo porta anche ad una maggiore regolarizzazione dei parametri dell'apprendimento, migliorando la robustezza all'overfitting.

Questa tecnica ha permesso quindi la progettazione di reti sempre più profonde, ma non elimina tutti i problemi collegati all'estrema profondità dell'architettura (persiste ad esempio il problema di degradazione descritto nel precedente paragrafo).

\subsection{Residual blocks}
\label{residualBlock}
Il principale contributo di ResNet nell'ambito del deep learning è sicuramente l'introduzione dei cosiddetti \textit{residual blocks} e delle \textit{shortcut connections} ("scorciatoie") per risolvere il problema di degradazione in precedenza descritto. In fig. \ref{fig:residualBlock} è mostrato uno schema generale dell'oggetto in esame.

\begin{figure}[h!]
\centering
\includegraphics[width=0.6\textwidth]{residualBlock.png}
\caption{\textit{Residual block} con \textit{shortcut identity mapping}. Si noti che i particolari layer di questo blocco sono arbitrari.}
\label{fig:residualBlock}
\end{figure}

Invece di sperare che un dato blocco di layer non lineari contigui impari ad approssimare una certa funzione $\mathcal{H}(\mathbf{x})$ desiderata, si decide di creare una "scorciatoia" (propriamente \textit{shortcut connection}) che connette l'input $\mathbf{x}$ e l'output del blocco (vd. fig. \ref{fig:residualBlock}); in questo modo il blocco deve imparare ad approssimare non più la funzione desiderata $\mathcal{H}(\mathbf{x})$ ma una funzione residuale $\mathcal{F}(\mathbf{x})\coloneqq\mathcal{H}(\mathbf{x})-\mathbf{x}$. In definitiva il blocco addestrato darà in output la funzione $\mathcal{F}(\mathbf{x})+\mathbf{x}$.\\

Per introdurre la motivazione che spinge alla progettazione di questo particolare blocco, facciamo il seguente esperimento.
Consideriamo una generica rete neurale con $n$ layer. Costruiamo una rete neurale con $n+k$ layer che ha come layer iniziali gli $n$ layer della prima rete e i rimanenti $k$ layer approssimano funzioni identità ($\mathcal{F}(\mathbf{x})=\mathbf{x}$). L'esistenza di una seconda rete più profonda della prima ma con le stesse prestazioni indica che una rete più profonda dovrebbe avere un \textit{training error} più basso o almeno uguale a quella meno profonda. Gli esperimenti in \cite{resnet} hanno mostrato tuttavia che non conosciamo nessun algoritmo di ottimizzazione che permetta almeno di eguagliare il \textit{training error} della prima rete addestrando la seconda rete descritta (almeno in un tempo non lungo). Questa è una delle forme in cui si manifesta il \textit{degradation problem} discusso in precedenza.

L'idea dei \textit{residual blocks}, peraltro non nuova e già utilizzata in precedenza (ad es. \cite{highway}), è basata sull'ipotesi degli autori - corroborata dall'esperimento di cui sopra - che le reti neurali abbiano difficoltà ad approssimare attraverso i loro tanti layer non lineari una funzione lineare (come appunto l'identità $\mathcal{f}(\mathbf{x})=\mathbf{x}$); in altre parole, viene ipotizzato che sia più facile per un blocco di layer imparare la funzione residua rispetto alla funzione originale. Ecco perché si decide di fornire direttamente l'input in uscita con una \textit{shortcut connection}, evitando al blocco lo sforzo di dover imparare a mappare un'identità.\\

Le straordinarie prestazioni raggiunte dalle reti ResNet estremamente profonde confermano che l'intuizione degli autori era corretta. Senza aver aggiunto complessità computazionale (escludendo le trascurabili somme dovute alle \textit{shortcut connections}) resta così risolto il problema di degradazione della \textit{training accuracy}.

\subsection{Architettura di ResNet-18}
ResNet-18 è una rete neurale convoluzionale profonda addestrata sul dataset ImageNet (par. \ref{imagenet}). La rete accetta in input immagini $224\times 224$ ed è composta da 18 layer parametrizzati, di cui un layer convoluzionale iniziale con filtro $7\times 7$ (seguito da batch normalization, ReLU e max pooling), altri 16 layer convoluzionali $3\times 3$ raccolti a due a due in 8 \textit{residual blocks} (descritti di seguito) e infine  un layer completamente connesso (preceduto da ReLU e average pooling) dalle cui 1000 attivazioni si calcola la distribuzione di probabilità per le 1000 classi di ImageNet, per mezzo della funzione softmax.
Il particolare \textit{residual block} adoperato in ResNet-18 è del tipo mostrato in figura \ref{fig:residualBlockResnet18}

\begin{figure}[H]
\centering
\includegraphics[width=0.4\textwidth]{residualBlockResnet18.png}
\caption{Uno degli otto \textit{residual blocks} di ResNet-18}
\label{fig:residualBlockResnet18}
\end{figure}

Il ramo che ospita i layer non lineari si compone delle seguenti operazioni:

\begin{itemize}
\item Convoluzione $3\times 3$\footnote{il numero di filtri di convoluzione, e quindi la profondità del volume di output, è riportata per ciascun \textit{residual block} in fig. \ref{fig:architetturaResnet18} e in fig. \ref{fig:confrontoResnet}}
\item Batch Normalization
\item ReLU
\item Convoluzione $3\times 3$
\item Batch Normalization
\end{itemize}

Inoltre, il ramo che realizza la \textit{shortcut connection} si compone eventualmente di una convoluzione $1\times 1$ seguita da batch normalization per garantire che il volume di attivazioni che attraversa la "scorciatoia" abbia dimensioni confrontabili con il volume uscente dal ramo che ospita i layer non lineari. Questa eventualità è rappresentata in fig. \ref{fig:architetturaResnet50} con un arco tratteggiato.\\

L'architettura di ResNet-18 è riportata schematicamente nella figura \ref{fig:architetturaResnet18} e in forma tabellare, in confronto con ResNet-50, in fig. \ref{fig:confrontoResnet} nel prossimo sottoparagrafo.

\subsection{Architettura di ResNet-50}
ResNet-50 è una variante più profonda di ResNet-18. Come la sua omologa meno profonda, questa rete accetta in input immagini $224\times 224$; essa è composta da 50 layer parametrizzati, di cui un layer convoluzionale iniziale con filtro $7\times 7$ (seguito da \textit{batch normalization}, ReLU e max pooling), altri 48 layer convoluzionali di varie dimensioni raccolti a gruppi di tre in 16 \textit{residual blocks} (descritti di seguito) e infine  un layer completamente connesso con le usuali 1000 attivazioni.
Il particolare \textit{residual block} adoperato in ResNet-50 è del tipo mostrato in figura \ref{fig:residualBlockResnet50}

\begin{figure}[H]
\centering
\includegraphics[width=0.4\textwidth]{residualBlockResnet50.png}
\caption{Uno dei sedici \textit{residual blocks} di ResNet-50}
\label{fig:residualBlockResnet50}
\end{figure}

Il ramo che ospita i layer non lineari si compone delle seguenti operazioni:

\begin{itemize}
\item Convoluzione $1\times 1$
\item Batch Normalization
\item ReLU
\item Convoluzione $3\times 3$
\item Batch Normalization
\item ReLU
\item Convoluzione $1\times 1$
\item Batch Normalization
\end{itemize}

Inoltre, il ramo che realizza la \textit{shortcut connection} si compone eventualmente di una convoluzione $1\times 1$ seguita da batch normalization per garantire che il volume di attivazioni che attraversa la "scorciatoia" abbia dimensioni confrontabili con il volume uscente dal ramo che ospita i layer non lineari. Questa eventualità è rappresentata in fig. \ref{fig:architetturaResnet50} con un arco tratteggiato.\\

L'evidente differenza rispetto al \textit{residual block} di ResNet-18 è motivata dalla premura di dover mantenere basso il tempo necessario ad addestrare la rete. Le convoluzioni $3\times 3$ sono computazionalmente costose, pertanto in reti molto profonde non si possono prevedere tutti \textit{residual blocks} ciascuno operante due convoluzioni $3\times 3$. Si decide allora di modificare il \textit{residual block} usato: si decide di operare un'unica convoluzione $3\times 3$ per blocco, preceduta e seguita da una convoluzione $1\times 1$ (par. \ref{1x1conv}) per rispettivamente ridurre e ripristinare la profondità del volume di attivazioni su cui la convoluzione $3\times 3$ lavora, abbassando così il costo computazionale associato all'addestramento di ciascun \textit{residual block}.\\

L'architettura di ResNet-50 è riportata schematicamente nella figura \ref{fig:architetturaResnet50} e in forma tabellare, in confronto con quella di ResNet-18, in fig. \ref{fig:confrontoResnet} seguente.

\begin{figure}[h!]
\centering
\includegraphics[width=0.8\textwidth, height=\textheight, keepaspectratio]{architetturaResnet.png}
\caption{Le architetture di ResNet-18 e ResNet-50 a confronto}
\label{fig:confrontoResnet}
\end{figure}

\begin{figure}[h]
  \begin{minipage}[b]{0.475\textwidth}
  \centering
    \includegraphics[width=\textwidth, height=\textheight, keepaspectratio]{architetturaResnet18.png}
    \caption{Architettura di ResNet-18}
    \label{fig:architetturaResnet18}
  \end{minipage}
  \hfill
  \begin{minipage}[b]{0.475\textwidth}
  \centering
    \includegraphics[width=\textwidth, height=\textheight, keepaspectratio]{architetturaResnet50.png}
    \caption{Architettura di ResNet-50}
    \label{fig:architetturaResnet50}
  \end{minipage}
\end{figure}

\subsection{Data Augmentation}
\label{augmentationResnet}
Come d'uso, anche ResNet-18 e ResNet-50 adoperano una strategia di \textit{data augmentation} per ridurre l'overfitting al training set di ImageNet.
Ogni immagine è ridimensionata con il suo lato più corto estratto casualmente dall'intervallo $[256,480]$ e mantenendo l'\textit{aspect ratio}. Viene ritagliata una \textit{patch} (ritaglio) $224\times 224$ casuale dall'immagine così ridimensionata e ad ogni canale della patch viene sottratta la media dei valori R, G e B del dataset ImageNet (similmente ad AlexNet, par. \ref{augmentationAlexnet}). Il ritaglio è eventualmente capovolto orizzontalmente (50\% di probabilità)

\subsection{Addestramento di ResNet-18 e ResNet-50}
Le reti ResNet-18 e ResNet-50 sono state addestrate sul database ImageNet usando la discesa stocastica del gradiente con momento = 0.9, mini-batch = 256 e decadimento dei pesi (\textit{weight decay}) = 0.0001. Il \textit{learning rate} iniziale è 0.1, ed è soggetto ad una strategia di \textit{annealing} che lo riduce di un fattore 10 ogniqualvolta il \textit{training error} si stabilizza. Il numero massimo di iterazioni di addestramento è $6\times 10^5$. I pesi sono inizializzati secondo il metodo descritto in \cite{weightsResnet}, creato dagli stessi autori di ResNet. Dettagli più specifici sulla fase di addestramento di ResNet-18 e ResNet-50 possono essere trovati nel paper originale \cite{resnet}.

%% 3 - Esperimenti e risultati
\chapter{Esperimenti e risultati}\label{esperimenti}
In questo capitolo vengono presentati gli esperimenti condotti e si analizzano i risultati ottenuti.

\section{Descrizione dei dataset utilizzati}
\label{dataset}
Per la conduzione degli esperimenti sono stati adoperati due distinti dataset di immagini, di seguito descritti (TODO descrivere meglio)

\begin{itemize}
\item Il primo, usato per la fase di addestramento delle reti neurali, è una collezione di fotografie di tursiopi e grampi scattate tra il 2017 e il 2018 nel \textbf{Golfo di Taranto} (mar Ionio Settentrionale). Le fotografie sono state scattate e messe a disposizione dalla \textit{Jonian Dolphin Conservation}, un'associazione di ricerca scientifica privata finalizzata allo studio dei cetacei nel Mar Ionio Settentrionale. Il dataset contiene immagini acquisite in un'area di 14000 km\textsuperscript{2} percorsa su un catamarano e seguendo rotte prestabilite. Le macchine fotografiche utilizzate consistono di diversi corpi macchina (reflex) e diversi obiettivi ad essi associati.
Il dataset acquisito contiene in totale TODO immagini, suddivise in cartelle in base alla data degli scatti.
\item Il secondo, usato per testare le prestazioni dei classificatori binari precedentemente addestrati, consiste in un insieme di fotografie di tursiopi e grampi scattate nel mese di giugno 2018 nei pressi delle \textbf{Isole Azzorre} (Oceano Atlantico settentrionale) dall'associazione TODO.
Questo dataset contiene in totale 5793 immagini, anche questa volta suddivise in cartelle in base alla data degli scatti.
\end{itemize}

Entrambi i dataset contengono fotografie con una notevole risoluzione TODO. Tuttavia, prendendo visione delle immagini in ciascuno dei due dataset ci si rende subito conto che non tutte contengono pinne dorsali di cetacei: in alcune foto sono totalmente assenti, rendendo lo scatto totalmente privo di contenuto informativo per i biologi.
Anche laddove le pinne sono presenti, esse possono risultare sfocate o di bassa risoluzione se molto lontane. Infine, in tutte le fotografie sono inevitabilmente ritratti oggetti che non sono "informativi" ai fini dello studio delle sole pinne dorsali quali barche, persone, uccelli, terraferma (paesaggi), porzioni di cielo, boe, lo specchio d'acqua ma anche parti dei cetacei diversi dalla loro pinna dorsale, quali pinne caudali e laterali e la testa degli esemplari.

Si rende perciò necessario filtrare in qualche modo le sole immagini che raffigurano al loro interno pinne dorsali di cetacei; è utile inoltre ritagliare da queste immagini filtrate le sole regioni in cui è effettivamente presente una pinna (si vuole cioè isolare l'informazione utile dal resto dal dato originale). Per far questo, i due dataset sono stati rielaborati attraverso un \textbf{algoritmo di riconoscimento e cropping} delle pinne dorsali, di seguito descritto nelle sue caratteristiche salienti.

TODO spiegare che allora il classificatore binario classifica i ritagli e non le foto intere.

\section{CropFin v1: pre-processing e estrazione di feature dai dataset}
Per ritagliare ed estrarre dalle immagini originali le sole pinne dorsali, è stata utilizzata la routine \textit{CropFin v1} in linguaggio MATLAB sviluppata dall'ing. Gianvito Losapio \cite{gianvito} sulla base di un precedente lavoro dell'ing. Flavio Forenza \cite{flavio}.
Si può descrivere la routine in due fasi:
\begin{enumerate}
\item Segmentazione, filtraggio e ritaglio adattivo delle regioni delle immagini che possono verosimilmente contenere una pinna
\item Classificazione di ogni ritaglio ottenuto in due classi 'Pinna' e 'No Pinna', mediante una rete neurale artificiale creata \textit{ad-hoc}.
\end{enumerate}

La novità introdotta dal presente lavoro di tesi in merito al problema di estrazione delle pinne da un'immagine riguarda l'utilizzo di un metodo di classificazione basato sul \textit{transfer learning}. In pratica, quindi, la principale differenza rispetto a CropFin v1 è nella seconda fase della routine: la classificazione avviene con l'utilizzo non più di una rete artificiale creata da zero per il problema in analisi, bensì riutilizzando un insieme di reti neurali profonde addestrate su un diverso problema di classificazione e adattate al nostro task. Questo nuovo modello è descritto dettagliatamente nel par. \ref{esperimentoTL}.\\

Al fine di ottenere i ritagli delle pinne, la routine adoperata attua una sequenza di operazioni di preprocessing su ciascuna immagine per poi individuare ed infine ritagliare e salvare separatamente le sole porzioni di immagini che possono eventualmente contenere pinne. Tale sequenza è implementata mediante un ciclo \verb|for| che cicla su ogni immagine del dataset. Di seguito sono descritte sinteticamente le operazioni, nell'ordine in cui vengono applicate.

TODO inserire immagini in ogni subsection per visualizzare le funzioni utilizzate

\subsection{Descrizione di CropFin v1}
\label{descrizioneCropFin}
\subsection*{Ridimensionamento}
L'immagine è innanzitutto ridimensionata mediante la funzione MATLAB \verb|imresize|, al fine di ottenere una nuova immagine di risoluzione più bassa ($800\times 1200$).
Questa operazione di preprocessing è stata adottata per diminuire il costo computazionale delle operazioni successive.\footnote{La risoluzione di partenza delle immagini utilizzate è stata $6000\times 4000$, ottenendo una riduzione drastica di pixel del 96\%, da 24 milioni a 960 mila.}


\subsection*{CLAHE}
L'immagine ridimensionata è sottoposta ad una equalizzazione adattiva dell’istogramma a contrasto limitato (CLAHE). Questa operazione consente un miglioramento del contrasto dell'immagine, proprietà utile per migliorare l'efficienza della successiva operazione, la sogliatura dell'immagine secondo il metodo di Otsu.

\subsection*{Segmentazione}
L'immagine viene segmentata (cioè ogni pixel viene assegnato ad una di due classi: \textit{background} e \textit{foreground}) mediante il metodo di Otsu per la sogliatura automatica \cite{otsu}. Il metodo di Otsu viene usato nella sua versione classica a due livelli, rispetto agli istogrammi dei canali L e b. In particolare, viene applicata la sogliatura secondo Otsu separatamente al canale L e b, cioè calcolate le soglie di Otsu per i due canali, mediante la funzione \verb|multithresh|.
Avendo a disposizione tali soglie, l’ipotesi avanzata è che le pinne dorsali possono essere
isolate considerando le regioni di immagine che siano contemporaneamente:
\begin{itemize}
\item nella regione più scura del canale L, cioè a sinistra della soglia sul canale L
\item nella regione contenente il grigio del canale b, cioè a destra della soglia sul canale b
\end{itemize}
L'immagine segmentata (binarizzata) finale è ottenuta quindi annerendo quei pixel dell'immagine che non verificano le seguenti condizioni (o, equivalentemente, rendendo bianchi i pixel che le verificano)\footnote{L’idea alla base di questo approccio nasce da una precisa conoscenza del dominio e da
alcune ipotesi a priori riguardanti il contenuto delle immagini. In particolare, si suppone
che esse contengano generalmente solo mare (background) e cetacei (foreground),
e che queste due classi di oggetti contribuiscano alla creazione di due aree distinte e
separabili degli istogrammi dei canali L e b. Volendo dare un’interpretazione intuitiva,
si tratta di separare ciò che è grigio e più scuro da ciò che è blu e più chiaro. La scelta
dello spazio di colori Lab è motivata proprio dalla possibilità di automatizzare questo
tipo intuitivo di segmentazione.}
\begin{itemize}
\item valore della componente L minore della soglia di Otsu sul canale L
\item valore della componente b maggiore della soglia di Otsu sul canale b
\end{itemize}

\subsection*{Filtraggio delle regioni connesse}
L’immagine binaria ottenuta in seguito alla segmentazione viene filtrata in modo che siano scartate quelle regioni binarie connesse (anche dette \textit{blob}) che non presentano caratteristiche tali da poter rappresentare, verosimilmente, una pinna dorsale.
In particolare vengono utilizzati, consecutivamente due filtri:
\begin{enumerate}

\item il primo è applicato all'intera immagine binarizzata e serve a migliorare il risultato della sogliatura secondo Otsu.
Il filtro è configurato per mantenere, nell'ordine, le regioni connesse con le seguenti proprietà:
\begin{itemize}
\item prime 15 in ordine decrescente di \verb|Area| (n. di pixel che compongono la regione connessa)
\item \verb|Area| nel range \verb|[1600, 40000]|
\item \verb|Extent| nel range \verb|[-Inf, 0.55]| (rapporto tra \verb|Area| e il n. di pixel del più piccolo rettangolo che racchiude l'intera regione connessa, con i lati paralleli a due a due paralleli ai bordi dell'immagine)
\end{itemize}

\item il secondo è applicato come segue
\begin{enumerate}
\item Si ritaglia la foto originale in corrispondenza delle regioni mantenute in seguito all’applicazione del primo filtro, sulla base delle coordinate dei bounding box. Per ottenere ritagli leggermenti più larghi rispetto ai blob, al fine di non perdere eventuali parti della pinna erroneamente anneriti dopo la binarizzazione, ogni dimensione è aumentata del 20\%.
\item Si applica nuovamente, a ciascun ritaglio ottenuto, la sogliatura basata sul metodo di Otsu. In questo caso è omesso il miglioramento del contrasto mediante CLAHE prima del calcolo dei valori di soglia.
\item Si introduce a questo punto il secondo filtro, applicato alle regioni binarie ottenute per ciascun ritaglio. L’unico parametro utilizzato in questo caso è il seguente:
\begin{itemize}
\item \verb|Area| nel range \verb|[20000, 1000000]|
\end{itemize}
con lo scopo di isolare l’eventuale pinna (che rappresenta sicuramente la regione di area maggiore all’interno di ciascun ritaglio) in modo che possa essere sottoposta all’algoritmo di ritaglio adattivo, descritto nella sezione successiva.
\end{enumerate}
\end{enumerate}

\subsection*{Ritaglio adattivo}
Le regioni binarie mantenute in seguito alla fase di filtraggio sono sottoposte ad un algoritmo che consente di ottenere un ritaglio preciso in corrispondenza delle pinne.
Tale operazione si può definire "adattiva" nella misura in cui la regione di ritaglio è ottenuta a partire da precisi punti geometrici calcolati per ciascuna regione binaria.
Evitando di scendere nei dettagli implementativi e numerici (riportati nel par. 5.1 in \cite{gianvito}), si descrivono nell'ordine le operazioni effettuate sulle singole regioni binarie dall'algoritmo di ritaglio:
\begin{enumerate}
\item Si sottopone la regione binaria al riempimento dei cosiddetti \textit{holes}, cioè "buchi" anneriti racchiusi in una regione connessa, mediante la funzione \verb|imfill| con opzione \verb|'holes'|
\item Si individuano quattro punti di interesse; nell’ordine: punto più in alto, punto medio tra questo ed il centroide, punti di estrema sinistra e destra della regione connessa all’altezza del
punto medio
\item Si identifica un rettangolo che racchiuda i punti precedentemente trovati
\item Si trasla e si estende il rettangolo trovato in modo che contenga l’intera pinna, a seconda della sua orientazione.
\end{enumerate}

L'output di questa prima fase della routine sono i ritagli di quelle regioni dell'immagine originale che, verosimilmente, ritraggono una pinna dorsale. Questa ipotesi sul contenuto dei ritagli è sostenuta solamente sulla base del processo di segmentazione e filtraggio appena descritto.

La routine CropFin v1, nella sua prima fase di ritaglio adattivo, è stata applicata ai dataset degli scatti di Taranto e delle Azzorre. In tabella \ref{risultatiCrop} è riportato il numero di ritagli (\textit{crops}) prodotti da CropFin v1 con input i dataset sopracitati.

\begin{table}[h]

  \centering
  \begin{tabular}{c c c c c}
  \hline
  Dataset&N. foto&N. crop&di cui 'Pinna'&di cui 'No Pinna'\\
  \hline
  Taranto&10194&15228&4033&11195\\
  Azzorre&11290&20395& TODO & TODO \\
  \hline
  \end{tabular}
  
  \caption{Output della prima fase di CropFin v1}
  \label{risultatiCrop}

\end{table}

\subsection*{Classificazione mediante rete ad hoc}
È evidente da una rapida ispezione dell'output che la quantità di regioni estratte che però non contengono pinne risulta, su larga scala, superiore a quello che contiene effettivamente pinne. Numericamente questo fatto è evidenziato in tab. \ref{risultatiCrop}, dopo una fase di etichettatura a mano dei ritagli prodotti, nelle classi 'Pinna' e 'No Pinna' (spiegata nel seguito del paragrafo).

Questa osservazione è ciò che primariamente motiva l’introduzione di una fase di classificazione finale in CropFin v1, che consenta di automatizzare completamente la procedura di object detection.

Come anticipato, in CropFin v1 si decide di effettuare la classificazione binaria 'Pinna'/'No Pinna' per mezzo di una rete neurale creata ad-hoc. In particolare, CropFin v1 prevede l'utilizzo di cinque classificatori binari, addestrati con la tecnica della \textit{5-fold cross-validation}\footnote{Si rimanda al par. \ref{crossval} per i dettagli implementativi della tecnica \textit{k-fold cross-validation}}
sui ritagli restituiti da CropFin v1 sul solo dataset con gli scatti di Taranto (descritto in \ref{dataset}).

Per consentire l'addestramento del classificatore si è reso necessario un lavoro di etichettatura manuale dei 15228 ritagli, attribuendo a ciascuno la classe 'Pinna' e 'No Pinna'. I risultati di questa etichettatura manuale sono presenti nella tab. \ref{risultatiCrop}.
Si precisa che, nella fase di etichettatura manuale, sono stati attribuiti alla classe 'Pinna' tutti e soli i ritagli contenenti una sola pinna in primo piano, intera o leggermente tagliata, escludendo invece quelli con pinne multiple e quelli con una presenza preponderante del dorso dei delfini. I ritagli con tali caratteristiche, infatti, sono considerati maggiormente affidabili ai fini di una successiva foto-identificazione automatica delle pinne (ad esempio con il metodo basato sul metodo \textit{SIFT} sviluppato e descritto in \cite{emanuele}). Inoltre, questa scelta è stata anche motivata dall’intenzione di creare un "concetto univoco" utile a semplificare sia la selezione manuale sia l’apprendimento del classificatore.

Le cinque reti risultanti lavorano in sinergia per classificare ciascun ritaglio, utilizzando un metodo di \textit{major voting} (par. \ref{ensemble}) "ibrido", che rende la classificazione finale ternaria: se le classificazioni 'Pinna' prodotte dalle cinque reti
\begin{itemize}
\item sono maggiori o uguali a 4, la classificazione finale è 'Pinna'
\item sono pari a 2 o 3, la classificazione finale è 'Incerta'
\item sono minori o uguali a 1, la classificazione finale è 'No Pinna'
\end{itemize}
L'analisi delle prestazioni di questo tipo di classificazione è descritta nel par. 4.5 di \cite{gianvito}.\footnote{Qualora fosse necessario attenersi a un problema di classificazione strettamente binario, ad esempio per effettuare un confronto con altri metodi di classificazione binaria per il problema in esame, si possono ad esempio ricondurre i ritagli di classe 'Incerta' alla classe 'No Pinna'. Questa è la scelta effettuata nel par. TODO per il confronto}



TODO dopo quando faccio il confronto con la rete di Gianvito sui 500 campioni devo scrivere "Per consentire il confronto del classificatore \textit{ensemble} di CropFin descritto nel par. \ref{classCropFin} e il classificatore \textit{ensemble} creato nel presente lavoro di tesi si è scelto di ritenere le pinne 'Incerta' come 'No Pinna'."

\section{Classificazione mediante CNN e Transfer Learning}
\label{esperimentoTL}
Nel par. \ref{transferlearning} sono stati descritti molteplici motivazioni per le quali per risolvere un problema di classificazione (in particolare di \textit{image captioning}) può essere meglio usare la tecnica del \textit{transfer learning}, adattando al task in esame una rete neurale pre-addestrata piuttosto che creare una rete da zero.
Il nucleo principale di questo lavoro di tesi è quindi dedicato alla creazione di un nuovo modello di classificazione, basato su \textit{transfer learning}, che possa migliorare la fase di classificazione di CropFin v1. Questi "miglioramenti" sono da valutare con rigore ingegneristico sulla base di alcuni parametri, che consentono un confronto di prestazioni con il classificatore di CropFin v1; l'analisi delle prestazioni e quindi il confronto è svolto nel par. \ref{prestazioni}.

\subsection{Creazione del dataset}
\label{creazioneDataset}
Il dataset utilizzato per l'addestramento del nuovo classificatore binario è lo stesso usato in \cite{gianvito} per l'addestramento del classificatore di CropFin v1 composto dai ritagli, opportunamente etichettati a mano, prodotti a partire dagli scatti collezionati nel Golfo di Taranto.

\subsection{Addestramento}
Sono state riutilizzate ed adattate mediante la tecnica del \textit{transfer learning} quattro reti neurali convoluzionali (\textit{CNN}) sviluppate nell'ambito della \textit{ImageNet Large Scale Visual Recognition Challenge (ILSVRC)} ed addestrate sul dataset ImageNet (par. \ref{imagenet}). Esse sono di seguito elencate e, per ciascuna, ne viene motivata la scelta.

\begin{itemize}

\item \textbf{AlexNet} (par. \ref{alexnet})\\
La sua vittoria nella \textit{ILSVRC 2012} con un grado di accuratezza del 16.4\% ha di fatto dimostrato alla comunità scientifica la straordinaria efficienza delle reti neurali convoluzionali nell'ambito dei problemi di \textit{computer vision}.

\item \textbf{GoogLeNet} (par. \ref{googlenet})\\
Grazie all'introduzione del modulo \textit{Inception}, GoogLeNet è una rete profonda ma incredibilmente leggera e semplice da addestrare, se paragonata alle precedenti reti fino ad allora esistenti (tra tutte, AlexNet).

\item \textbf{ResNet-18} (par. \ref{resnet})\\
ResNet rappresenta lo stato dell'arte nell'ambito delle reti neurali convoluzionali; l'introduzione dei \textit{residual blocks} ha permesso di avere reti con un grandissimo numero di layer, attenuando di molto i problemi legati all'estrema profondità dell'architettura.

\item \textbf{ResNet-50} (par. \ref{resnet})\\
Una variante di ResNet-18, più profonda e con migliori prestazioni sul dataset \textit{ImageNet}.

\end{itemize}





%% 4 - Conclusioni
\chapter{Conclusioni e sviluppi futuri}
Nel presente lavoro di tesi è stato affrontato un problema di \textit{object detection} con oggetto il rilevamento di pinne dorsali di cetacei in una collezione di immagini, adoperando tecniche di \textit{computer vision} e di \textit{deep learning}.
In particolare, si è deciso di utilizzare il metodo del \textit{transfer learning} per migliorare le prestazioni registrate dagli algoritmi CropFin v1 e v2 sviluppati in \cite{gianvito} che per primi hanno risolto il task in esame, e che costituiscono il punto di partenza degli esperimenti effettuati in questa tesi.\\

L'output degli esperimenti è stato un nuovo classificatore \textit{ensemble}, composto da tre reti neurali convoluzionali pre-addestrate di diversa profondità e capacità, adattate a risolvere il problema della classificazione binaria 'Pinna'/'No Pinna', a partire dai ritagli delle eventuali pinne di un'immagine prodotti dalla prima parte della routine CropFin v1. Questo classificatore si innesta nella routine CropFin v1 andando a sostituire il classificatore nativo, basato sull'utilizzo di una CNN \textit{from scratch}, registrando un netto miglioramento di prestazioni nella fase di classificazione di ritagli (test error 97.2\% e specificity 98.1\% su una collezione di ritagli mai vista dall'ensemble, contro il 92\% e 95\% del classificatore nativo). Il raggiungimento di tali prestazioni sono il risultato di diversi fattori: in primo luogo, l'alta capacità di rappresentazione e astrazione dei concetti raggiunte dalle tre reti utilizzate; inoltre, il dominio ristretto del problema in esame e la disponibilità di immagini per l'addestramento in numerose condizioni di scatto, che garantiscono buona generalizzazione al nostro modello di classificazione.\\

Il tentativo di un nuovo approccio al problema con la tecnica del \textit{transfer learning}, suggerito dai recenti successi registrati grazie all'utilizzo di questa tecnica (\cite{tl1}, \cite{tl2}, \cite{tl3}, tra quelli più vicini al problema in esame), è pienamente giustificato dagli ottimi risultati ottenuti in termini di prestazioni.\\

Il problema di rilevamento delle pinne di cetacei all'interno di un'immagine si inserisce in un più ampio problema di \textit{object detection}, che ha come oggetto la foto-identificazione automatica dei delfini avvistati durante le campagne di avvistamento in mare aperto. In previsione di un utilizzo dei ritagli classificati dal nostro ensemble di reti da parte di routine che compiano appunto questo foto-riconoscimento (\textit{SPIR} \cite{maglietta} e \textit{SPIR v2}\cite{emanuele}) è stato fondamentale ridurre quanto più possibile il valore di specificity (che dà una misura di quanto i falsi positivi "inquinano" i ritagli invece correttamente classificati come 'Pinna') per dare in input a queste routine solo ritagli che raffigurino effettivamente una pinna.\\

La metodologia di \textit{object detection} introdotta può essere il punto di partenza per interessanti sviluppi futuri.\\

Si può pensare di riutilizzare il classificatore mediante \textit{transfer learning} per risolvere il task di riconoscimento delle pinne dorsali anche per quelle specie marine il cui studio da parte dei biologi preveda il foto-riconoscimento degli esemplari a partire dalla loro pinna dorsale, quali orche (\textit{Orcinus orca}) \cite{orche} e squali \cite{squali}.\\

Gli algoritmi di \textit{photo-ID} dei cetacei a partire dalla loro pinna, come quello descritto in \cite{emanuele}, hanno il difetto di assegnare \emph{sempre} una "identità" probabile ad un ritaglio (prodotto ad esempio come output di CropFin v1), anche nel caso in cui esso non rappresenti davvero una pinna. In questo senso, cercare di migliorare il più possibile la specificity del classificatore su un generico test set diventa di primaria importanza. Ma non solo: anche quei ritagli che sono correttamente etichettati come pinne possono in vario modo essere non idonei al foto-riconoscimento automatizzato per mezzo di un algoritmo (ad esempio perché la pinna risulta troppo piccola, sfocata, disturbata da schizzi d'acqua: in generale sono problematici tutti quei ritagli in cui le \textit{feature} che permettono un'identificazione univoca dell'esemplare, come i noti graffi dei grampi o il bordo delle pinne dei tursiopi, sono scarsamente evidenti e pertanto inaffidabili).

Per risolvere questa criticità si può pensare di agire ulteriormente sul set dei ritagli da classificare, escludendo quelli ritenuti dalla macchina "non idonei" al successivo foto-riconoscimento, magari mediante l'uso di tecniche di \textit{computer vision} che permettano di filtrare ulteriormente i ritagli. Ad esempio, sarebbe utile riconoscere (ed escludere) quando una pinna è troppo sfocata (la macchina può verificare che il gradiente ai bordi della pinna è poco "ripido" e porta dal grigio della pinna al blu-verde dell'acqua senza significativa soluzione di continuità).
In alternativa, si può tentare un approccio che prevede il ri-addestramento del classificatore ensemble proposto in questa tesi (ancora una volta con la tecnica del \textit{transfer learning}) presentandogli come training set una collezione di ritagli divisi tra 'Idoneo' e 'Non idoneo' al foto-riconoscimento.\\



%% 5 - Sorgenti
\chapter{Listato del codice sorgente}
\label{sorgenti}


\section{Funzioni utili per il transfer learning}

\subsection{createLgraphUsingConnections.m}
\lstinputlisting[language=Matlab]{sorgenti/createLgraphUsingConnections.m}
\vspace{3mm}

\subsection{findLayersToReplace.m}
\lstinputlisting[language=Matlab]{sorgenti/findLayersToReplace.m}
\vspace{3mm}

\subsection{freezeWeights.m}
\lstinputlisting[language=Matlab]{sorgenti/freezeWeights.m}
\vspace{3mm}

\subsection{plotConfusionMatrix.m}
\lstinputlisting[language=Matlab]{sorgenti/plotConfusionMatrix.m}


\section{Ri-addestramento CNN}

\subsection{AlexNet}
\lstinputlisting[language=Matlab]{sorgenti/training_alexnet.m}
\vspace{3mm}

\subsection{GoogLeNet}
\lstinputlisting[language=Matlab]{sorgenti/training_googlenet.m}
\vspace{3mm}

\subsection{ResNet-18}
\lstinputlisting[language=Matlab]{sorgenti/training_resnet18.m}
\vspace{3mm}

\subsection{ResNet-50}
\lstinputlisting[language=Matlab]{sorgenti/training_resnet50.m}

\section{Test sul dataset delle Azzorre}

\subsection{Singoli classificatori}
\lstinputlisting[language=Matlab]{sorgenti/test_azzorre.m}
\vspace{3mm}

\subsection{Classificatore ensemble}
\lstinputlisting[language=Matlab]{sorgenti/major_voting.m}






%% BIBLIOGRAFIA
\backmatter

\nocite{*}
\printbibliography[heading=bibintoc]		%stampa la bibliografia alla fine e la aggiunge all'indice generale


\end{document}
