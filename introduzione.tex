\chapter{Introduzione}
\label{introduzione}
%\addcontentsline{toc}{chapter}{Introduzione}
La foto-identificazione è una tecnica largamente impiegata per l’identificazione dei singoli individui a partire da una o più immagini. Il principale vantaggio di questa tecnica è la sua non invasività che la rende particolarmente utile per studiare sia la dinamicità che i movimenti di ogni specie. Tale metodologia risulta essere uno strumento affidabile quando viene applicato nella comprensione dei comportamenti dei cetacei (migrazioni e spostamenti). Tra i delfini, vi sono due specie più adatte a tali studi.  Il primo delfino riguarda la specie “Tursiops truncatus” (tursiope) o delfino dal naso a bottiglia, mentre la seconda specie riguarda la specie “Grampus griseus” (Grampo, o delfino di Risso), avente numerose cicatrici su tutto il corpo, entrambi appartenenti alla famiglia dei Delfinidi.[1]

\section{Problema e obiettivi}
L'obiettivo principale del presente lavoro di tesi è stato quello di creare un sistema per il rilevamento automatico della presenza di cetacei all'interno di uno scatto fotografico.

Lo scopo finale è facilitare lo studio dei cetacei, attorno al quale si riunisce grande interesse scientifico (par. \ref{TODO}), incentivando l'utilizzo di tecniche non invasive basate su algoritmi innovativi e grandi disponibilità di dati. Uno dei principali metodi di studio non-invasivi dei cetacei è la foto-identificazione degli esemplari (par. \ref{fotoIdentificazione}). Il presente lavoro di tesi rappresenta un passo avanti verso la completa automatizzazione del processo di foto-identificazione degli esemplari incontrati durante le campagne di avvistamento, fornendo un miglioramento nelle prestazioni di una routine (CropFin v1, par. \ref{cropFin}) che può aiutare i biologi nel successivo lavoro di foto-identificazione.

Il problema affrontato TODO