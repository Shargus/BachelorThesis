\chapter*{Introduzione}
\addcontentsline{toc}{chapter}{Introduzione}
La foto-identificazione è una tecnica largamente impiegata per l’identificazione dei singoli individui a partire da una o più immagini. Il principale vantaggio di questa tecnica è la sua non invasività che la rende particolarmente utile per studiare sia la dinamicità che i movimenti di ogni specie. Tale metodologia risulta essere uno strumento affidabile quando viene applicato nella comprensione dei comportamenti dei cetacei (migrazioni e spostamenti). Tra i delfini, vi sono due specie più adatte a tali studi.  Il primo delfino riguarda la specie “Tursiops truncatus” (tursiope) o delfino dal naso a bottiglia, mentre la seconda specie riguarda la specie “Grampus griseus” (Grampo, o delfino di Risso), avente numerose cicatrici su tutto il corpo, entrambi appartenenti alla famiglia dei Delfinidi.[1]