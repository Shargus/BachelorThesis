%% pacchetti usati:
% babel, indent first, fancy headers, amssymb, amsmath, latexsym, geometry per le specifiche, parindent=1cm, frontespizio (!)

\usepackage[english,italian]{babel}
\usepackage[utf8]{inputenc}
\usepackage[T1]{fontenc}
\usepackage{indentfirst}
\usepackage{fancyhdr}
\usepackage[titletoc]{appendix} %appendici

% per far vedere le etichette, da togliere quando si deve stampare
%\usepackage{showkeys}

% per la matematica
\usepackage{amsmath}
\usepackage{latexsym}
\usepackage{amssymb}
\usepackage{mathtools}
\usepackage{stackengine}

% comandi utili
\newcommand{\eqdef}{\stackrel{\mathclap{\normalfont\mbox{def}}}{=}}
\newcommand\oast{\stackMath\mathbin{\stackinset{c}{0ex}{c}{0ex}{\ast}{\bigcirc}}}
\newcommand{\R}{\mathbb{R}}
\newcommand{\N}{\mathbb{N}}
\newcommand{\Z}{\mathbb{N}}
\newcommand{\ii}{(i)}
\newcommand{\XX}{\mathbf{X}}
\newcommand{\WW}{\mathbf{W}}
\newcommand{\bb}{\mathbf{b}}
\newcommand{\xii}{x_i}
\newcommand{\xjj}{x_j}
\newcommand{\abs}[1]{\left|#1\right|} %valore assoluto
\usepackage[top=3cm, bottom=3cm, left=3cm, right=3cm]{geometry} %ORIGINALE: l=3.5 r=2.5
\parindent=7mm
\usepackage[swapnames]{frontespizio}
\usepackage{changepage}
\newcommand*\justify{%
  \fontdimen2\font=0.4em% interword space
  \fontdimen3\font=0.2em% interword stretch
  \fontdimen4\font=0.1em% interword shrink
  \fontdimen7\font=0.1em% extra space
  \hyphenchar\font=`\-% allowing hyphenation
}

% per le immagini
\usepackage{graphicx}
\graphicspath{{img/}{img_esperimenti/}{img_appendice/}{img_esperimenti/foto_azzorre/}{img_esperimenti/foto_taranto/}{img_introduzione/}{img/googlenet/}{img/alexnet/}{img/resnet/}}
\usepackage{subcaption}
\usepackage{float}
\newcommand{\rulesep}{\unskip\ \vrule\ }

% per le tabelle
\usepackage{caption} %vale anche per le immagini
\captionsetup{aboveskip=10pt}
\captionsetup{belowskip=0pt}
\captionsetup[table]{position=bottom}
\usepackage{tabularx, booktabs}
\newcommand{\acapo}[1]{%
  \begin{tabular}{@{}c@{}}\strut#1\strut\end{tabular}%
}
\usepackage{adjustbox}

% per il codice sorgente
\usepackage{verbatim}
\usepackage{listings}
\usepackage[dvipsnames]{xcolor}
\definecolor{lightgray}{gray}{0.99}
\lstdefinestyle{mystyle}{
	backgroundcolor=\color{lightgray},   
	commentstyle=\color{Gray},
	%keywordstyle=\color{Blue},
	numberstyle=\tiny\color{gray},
	%stringstyle=\color{purple},
	basicstyle=\footnotesize,
	breakatwhitespace=false,         
	breaklines=true,                 
	captionpos=b,                    
	keepspaces=true,                 
	numbers=left,                    
	numbersep=5pt,                  
	showspaces=false,                
	showstringspaces=false,
	showtabs=false,                  
	tabsize=2
}
\lstset{style=mystyle}

% per l'epigrafe
\makeatletter
\newenvironment{chapquote}[2][2em]
  {\setlength{\@tempdima}{#1}%
   \def\chapquote@author{#2}%
   \parshape 1 \@tempdima \dimexpr\textwidth-2\@tempdima\relax%
   \itshape}
  {\par\normalfont\hfill--\ \chapquote@author\hspace*{\@tempdima}\par\bigskip}
\makeatother

% per la bibliografia
\usepackage[autostyle, italian=guillemets]{csquotes}
\usepackage[backend=biber, style=numeric-comp, babel=hyphen, sorting=none]{biblatex}
\usepackage{url}
\usepackage{fancyvrb}
\addbibresource{biblio_tesi.bib}
