\chapter{Conclusioni e sviluppi futuri}
\label{conclusioni}
Nel presente lavoro di tesi è stato affrontato un problema di \textit{object detection} con oggetto il rilevamento di pinne dorsali di cetacei in una collezione di immagini, adoperando tecniche di \textit{computer vision} e di \textit{deep learning}.
In particolare, si è deciso di utilizzare il metodo del \textit{transfer learning} per migliorare le prestazioni registrate dagli algoritmi CropFin v1 e v2 sviluppati in \cite{gianvito} che per primi hanno risolto il task in esame, e che costituiscono il punto di partenza degli esperimenti effettuati in questa tesi.\\

L'output degli esperimenti è stato un nuovo classificatore \textit{ensemble}, composto da tre reti neurali convoluzionali pre-addestrate di diversa profondità e capacità, adattate a risolvere il problema della classificazione binaria 'Pinna'/'No Pinna', a partire dai ritagli delle eventuali pinne di un'immagine prodotti dalla prima parte della routine CropFin v1. Questo classificatore si innesta nella routine CropFin v1 andando a sostituire il classificatore nativo, basato sull'utilizzo di una CNN \textit{from scratch}, registrando un netto miglioramento di prestazioni nella fase di classificazione di ritagli (test error 97.2\% e specificity 98.1\% su una collezione di ritagli mai vista dall'ensemble, contro il 92\% e 95\% del classificatore nativo). Il raggiungimento di tali prestazioni sono il risultato di diversi fattori: in primo luogo, l'alta capacità di rappresentazione e astrazione dei concetti raggiunte dalle tre reti utilizzate; inoltre, il dominio ristretto del problema in esame e la disponibilità di immagini per l'addestramento in numerose condizioni di scatto, che garantiscono buona generalizzazione al nostro modello di classificazione.\\

Il tentativo di un nuovo approccio al problema con la tecnica del \textit{transfer learning}, suggerito dai recenti successi registrati grazie all'utilizzo di questa tecnica (\cite{tl1}, \cite{tl2}, \cite{tl3}, tra quelli più vicini al problema in esame), è pienamente giustificato dagli ottimi risultati ottenuti in termini di prestazioni.\\

Il problema di rilevamento delle pinne di cetacei all'interno di un'immagine si inserisce in un più ampio problema di \textit{object detection}, che ha come oggetto la foto-identificazione automatica dei delfini avvistati durante le campagne di avvistamento in mare aperto. In previsione di un utilizzo dei ritagli classificati dall'ensemble di reti creato da parte di routine che compiano appunto questo foto-riconoscimento (\textit{SPIR} \cite{maglietta} e \textit{SPIR v2}\cite{emanuele}) è stato fondamentale ridurre quanto più possibile il valore di specificity (che dà una misura di quanto i falsi positivi "inquinano" i ritagli invece correttamente classificati come 'Pinna') per dare in input a queste routine solo ritagli che raffigurino effettivamente una pinna.\\

La metodologia di \textit{object detection} introdotta può essere il punto di partenza per interessanti lavori simili e sviluppi futuri.\\

Nell'ambito dei lavori affini a quello presentato in questo elaborato, si può pensare di riutilizzare il classificatore per risolvere il task di riconoscimento delle pinne dorsali anche di quelle specie marine il cui studio da parte dei biologi preveda il foto-riconoscimento degli esemplari a partire dalla loro pinna dorsale, quali orche (\textit{Orcinus orca}) \cite{orche} e squali \cite{squali}. Il ri-adattamento del classificatore è operato, ancora una volta, mediante la tecnica del \textit{transfer learning} applicata su di esso.\\

Un interessante sviluppo futuro del lavoro appena concluso può riguardare un ulteriore filtraggio del dataset dei ritagli da presentare all'algoritmo di foto-identificazione.
Gli algoritmi di \textit{photo-ID} dei cetacei a partire dalla loro pinna, come quello descritto in \cite{emanuele}, hanno il difetto di assegnare \emph{sempre} una "identità" probabile ad un ritaglio (prodotto ad esempio come output di CropFin v1), anche nel caso in cui esso non rappresenti davvero una pinna. In questo senso, cercare di migliorare il più possibile la specificity del classificatore su un generico test set diventa di primaria importanza. Ma non solo: anche quei ritagli che sono correttamente etichettati come pinne possono in vario modo essere non idonei al foto-riconoscimento automatizzato per mezzo di un algoritmo (ad esempio perché la pinna risulta troppo piccola, sfocata, disturbata da schizzi d'acqua: in generale sono problematici tutti quei ritagli in cui le \textit{feature} che permettono un'identificazione univoca dell'esemplare, come i noti graffi dei grampi o il bordo delle pinne dei tursiopi, sono scarsamente evidenti e pertanto inaffidabili).

Per risolvere questa criticità si può pensare di agire ulteriormente sul set dei ritagli da classificare, escludendo quelli ritenuti dalla macchina "non idonei" al successivo foto-riconoscimento, magari mediante l'uso di tecniche di \textit{computer vision} che permettano di filtrare ulteriormente i ritagli. Ad esempio, sarebbe utile riconoscere (ed escludere) quei ritagli in cui la pinna appare troppo sfocata (la macchina può verificare che il gradiente ai bordi della pinna è poco "ripido" e porta dal grigio della pinna al blu-verde dell'acqua senza significativa soluzione di continuità).
In alternativa, si può tentare un approccio che prevede il ri-addestramento del classificatore ensemble proposto in questa tesi (ancora una volta con la tecnica del \textit{transfer learning}) presentandogli come training set una collezione di ritagli divisi tra 'Idoneo' e 'Non idoneo' al foto-riconoscimento.\\

